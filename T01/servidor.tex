\section{Módulo Servidor}
\begin{interfaz}{\subsection{Interfaz}}
  \seExplica{Servidor}{servidor}\\\\
  \usa{Variante, Cola, Letra, Ocurrencia, Juego, Notificación}\par\noindent
  \begin{operaciones}
    \InterfazFuncion{nuevoServidor}
    {\In{k}{nat}, \In{v}{variante}, \In{r}{cola(letra)}}{servidor}
    [$tama\tilde{n}o(r)\geq tama\tilde{n}oTablero(v)*tama\tilde{n}oTablero(v)+k*\#fichas(v)$]
    {$res \igobs nuevoServidor(k,v,r)$}
    [$O(N^{2}+|\Sigma|K+FK)$]
    [Dada una cantidad de jugadores y una variante de juego, se inicia un nuevo servidor y una nueva partida de juego.]
    [
    $v:\texttt{variante}$ tiene referencia \textbf{no} modificable.\\
    \-\hspace{5em}$r:\texttt{cola(letra)}$ tiene referencia modificable.
    ]\\

    \noindent\InterfazFuncion{conectar}
    {\Inout{s}{servidor}}{}
    [$\lnot empez\acute{o}?(s) \land s\igobs S_{0}$]
    {$s\igobs conectarCliente(S_{0})$}
    [$O(1)$]
    [Conecta un cliente a un servidor.]\\

    \noindent\InterfazFuncion{consultar}
    {\Inout{s}{servidor}, \In{cid}{nat}}{cola(notif)}
    [$cid\leq \#conectados(s) \land s\igobs S_{0}$]
    {$s\igobs consultar(S_{0},cid)\land res\igobs notificaciones(S_{0},cid)$}
    [$O(n)$, donde $n$ es la cantidad de mensajes en la cola de dicho cliente.]
    [Consulta la cola de notificaciones de un cliente (lo cual vacía dicha cola).]
    [Se devuelve $res:\texttt{cola(notif)}$ como referencia \textbf{no} modificable.]\\

    \noindent\InterfazFuncion{recibir}
    {\Inout{s}{servidor}, \In{cid}{nat}, \In{o}{ocurrencia}}{}
    [$cid\leq \#conectados(s) \land s\igobs S_{0}$]
    {$s\igobs recibirMensaje(S_{0},cid,o)$}
    [$O(m\cdot\Lmax^{2}+|\Sigma|K+FK)$]
    [Recibe un mensaje de un cliente.]
    [Se pasa $o:\texttt{ocurrencia}$ como referencia \textbf{no} modificable.]\\

    \noindent\InterfazFuncion{clientesEsperados}
    {\In{s}{servidor}}{nat}
    [true]
    {$res\igobs \#esperados(s)$}
    [$O(1)$]
    [Obtiene el número de clientes esperados.]\\

    \noindent\InterfazFuncion{clientesConectados}
    {\In{s}{servidor}}{nat}
    [true]
    {$res\igobs \#conectados(s)$}
    [$O(1)$]
    [Obtiene el número de clientes conectados.]\\

    \noindent\InterfazFuncion{partida}
    {\In{s}{servidor}}{juego}
    [$empez\acute{o}?(s)$]
    {$res\igobs juego(s)$}
    [$O(1)$]
    [Obtiene el juego que se está jugando en el servidor.]
    [Se devuelve $res:\texttt{juego}$ como referencia \textbf{no} modificable.]\\

    \noindent\InterfazFuncion{empezó?}
    {\In{s}{servidor}}{bool}
    [true]
    {$res\igobs empez\acute{o}?(s)$}
    [$O(1)$]
    [Determina si la partida empezó.]
  \end{operaciones}
\end{interfaz}

\subsection{Implementación}

\subsubsection{Representación}

\begin{Estructura}{servidor}[servidor\_estr]
\begin{Tupla}[servidor\_estr]
\begin{adjustwidth}{3em}{}\ \
  \tupItem{juego}{juego}\\
  \tupItem{jugadoresConectados}{nat}\\
  \tupItem{jugadoresEsperados}{nat}\\
  \tupItem{notificaciones}{array\_dimensionable(cola(notif))}
\end{adjustwidth}\ \ \ \ \ \ \
\end{Tupla}
\end{Estructura}

\subsubsection{Invariante de Representación}
\Rep[servidor\_estr]{$
    \#jugadores(e.juego) = e.jugadoresEsperados\land\\
    e.jugadoresConectados \leq e.jugadoresEsperados\land\\
    tam(e.notificaciones) = e.jugadoresEsperados
 $
 }

\subsubsection{Función de Abstracción}
\Abs[servidor\_estr]{servidor}[e]{S}{$
  e.jugadoresEsperados\igobs \#esperados(S)\land\\
  e.jugadoresConectados\igobs \#conectados(S)\land\\
  e.juego\igobs juego(S)\land\\
  \langle variante(e.juego), repositorio(e.juego)\rangle\igobs configuracion(S)\land\\
  (\forall i:\texttt{nat})(i<\#esperados(S)\impluego
  e.notificaciones[i] \igobs notificaciones(S, i))
  $
}

\subsubsection{Algoritmos}

\begin{algorithm}[H]
    \defun{iNuevoServidor}{\In{k}{nat}, \In{v}{variante}, \In{r}{cola(letra)}}{servidor\_estr}
    \begin{algorithmic}[1]
        \State\asignar{res.juego}{\textsc{nuevoJuego}(k,v,r)}\Comment{$O(N^{2}+|\Sigma|K+FK)$}
        \State\asignar{res.jugadoresConectados}{0}
        \State\asignar{res.jugadoresEsperados}{k}
        \State\asignar{res.notificaciones}{\textsc{CrearArreglo}(k)}
        \For{$notif\in res.notificaciones$}
            \State\asignar{notif}{\textsc{Vacía}()}
        \EndFor
        \State\Return$res$
    \end{algorithmic}
\end{algorithm}

\begin{algorithm}[H]
    \defunv{iConectar}{\Inout{s}{servidor\_estr}}
    \begin{algorithmic}[1]
        \State$\textsc{Encolar}(s.notificaciones[s.jugadoresConectados],\textsc{IdCliente}(s.jugadoresConectados))$
        \State$s.jugadoresConectados++$
        \If{$\textsc{empezó?}(s)$}
            \For{$notif\in s.notificaciones$}
                \State$\textsc{Encolar}(notif,\textsc{Empezar}(\textsc{tamañoTablero}(\textsc{variante}(s.juego)))$
                \State$\textsc{Encolar}(notif,\textsc{TurnoDe}(0))$
            \EndFor
        \EndIf
    \end{algorithmic}
\end{algorithm}

\begin{algorithm}[H]
    \defun{iConsultar}{\Inout{s}{servidor\_estr}, \In{cid}{nat}}{cola(notif)}
    \begin{algorithmic}[1]
        \State\asignar{res}{\textsc{Copiar}(s.notificaciones[cid])}\Comment{$\Theta(\sum_{i=1}^{n}copy(c[i]))=\Theta(\sum_{i=1}^{n}1)=\Theta(n)\in O(n)$, con $c=s.notificaciones[cid]$}
        \State\asignar{s.notificaciones[cid]}{\textsc{Vacía}()}\Comment$O(1)$
        \State\Return$res$
    \end{algorithmic}
\end{algorithm}

\begin{algorithm}[H]
    \defunv{iRecibir}{\Inout{s}{servidor\_estr}, \In{cid}{nat}, \In{o}{ocurrencia}}
    \begin{algorithmic}[1]
        \If{$\textsc{empezó?}(s)\yluego\textsc{turno}(s.juego)=cid \yluego \textsc{jugadaVálida?}(s.juego, o)$}
                    \State\asignar{\#fichasRepuestas}{\textsc{fichasPorJugador}(s.configuraci\acute{o}n.variante)-\#o}
            \State\asignar{repoViejo}{copy(s.configuraci\acute{o}n.repositorio)}
            \State\asignar{repuestos}{\textsc{CrearArreglo}(\textsc{dom}())}\Comment{Asumimos que \texttt{multiconj(letra)} es un Arreglo Dimensionable del tamaño de dominio de Letra}
            \For{$cant\in repuestos$}
                \State\asignar{cant}{0}
            \EndFor
            \For{$1\dots\#fichasRepuestas$}
                \State\asignar{ficha}{\textsc{Proximo}(repoViejo)}
                \State$repuestos[\textsc{ord}(ficha)]++$
                \iffalse
                \If{$\textsc{Definido?}(repuestos,ficha)$}
                    \State$\textsc{Definir}(repuestos,ficha,\textsc{Significado}(repuestos,ficha)+1)$
                \Else
                    \State$\textsc{Definir}(repuestos,ficha,1)$
                \EndIf
                \fi
                \State$\textsc{Desencolar}(repoViejo)$
            \EndFor
            \State\asignar{puntajeViejo}{\textsc{puntaje}(s.juego,cid)}
            \State$\textsc{ubicar}(s.juego,o)$
            \State\asignar{puntajeNuevo}{\textsc{puntaje}(s.juego,cid)}
            \For{$0\leq i<tam(s.notificaciones)$}
                \State\asignar{notif}{s.notificaciones[i]}
                \State$\textsc{Encolar}(notif,\textsc{Ubicar}(cid,o)$\Comment{\textsc{Ubicar} se refiere al item de notificación}
                \State$\textsc{Encolar}(notif,\textsc{SumaPuntos}(cid,puntajeNuevo-puntajeViejo)$
                \If{$i=cid$}
                    \State$\textsc{Encolar}(notif,\textsc{Reponer}(repuestos))$
                \EndIf
                \State$\textsc{Encolar}(notif,\textsc{TurnoDe}(\textsc{turno}(s.juego))$
            \EndFor
            \iffalse
            \If{\textsc{Cardinal}(o)$>0$}
                \State\asignar{nuevasLetras}{\textsc{CrearArreglo}(\textsc{Cardinal}(o))}
                \For{$1\leq i < $\textsc{ Cardinal}(o)}
                    \State\asignar{letra}{\textsc{Desapilar}(s.juego.repositorio)}
                    \State\textsc{AgregarAtras}($nuevasLetras$, $letra$)
                \EndFor
                \For{$letra\in nuevasLetras$}
                    \State{\textsc{Apilar}(letra, s.juego.repositorio)}
                \EndFor
                \State\textsc{Encolar}(\textsc{Reponer}($nuevasLetras$), $s.notificaciones[cid]$)
            \EndIf
            \State\textsc{ubicar}(s.juego, o)
            \For{$1\leq i < s.jugadoresConectados$}
               \State\textsc{Encolar}(\textsc{Ubicar}($juego.variante.n$), $s.notificaciones[i]$)
               \State\textsc{Encolar}(\textsc{TurnoDe}(\textsc{Turno}($s.juego$)), $s.notificaciones[i]$)
            \EndFor
            \fi
        \Else
            \State$\textsc{Encolar}(s.notificaciones[cid], \textsc{Mal})$
        \EndIf
    \end{algorithmic}
\end{algorithm}

\begin{algorithm}[H]
    \defun{iClientesEsperados}{\In{s}{servidor\_estr}}{nat}
    \begin{algorithmic}[1]
        \State\Return$s.jugadoresEsperados$
    \end{algorithmic}
\end{algorithm}

\begin{algorithm}[H]
    \defun{iClientesConectados}{\In{s}{servidor\_estr}}{nat}
    \begin{algorithmic}[1]
        \State\Return$s.jugadoresConectados$
    \end{algorithmic}
\end{algorithm}

\begin{algorithm}[H]
    \defun{iPartida}{\In{s}{servidor\_estr}}{juego}
    \begin{algorithmic}[1]
        \State\Return$s.juego$
    \end{algorithmic}
\end{algorithm}

\begin{algorithm}[H]
    \defun{iEmpezó?}{\In{s}{servidor\_estr}}{bool}
    \begin{algorithmic}[1]
        \State\Return$s.jugadoresEsperados=s.jugadoresConectados$
    \end{algorithmic}
\end{algorithm}
\newpage

%%% Local Variables:
%%% mode: latex
%%% TeX-master: "main"
%%% End:
