\newcommand{\seExplica}[2]{
\textbf{se explica con}: \NombreFuncion{#1}\par
\textbf{géneros}: \TipoVariable{#2}\par
}

\newcommand{\usa}[1] {
\textbf{usa}: \NombreFuncion{#1}\par
}

\newenvironment{operaciones}{
  \textbf{operaciones}:\par
  \setlength\parindent{3em}
}

\newcommand{\asignar}[2]{$#1 \gets #2$}

\section{Módulo Servidor}
\subsection{Interfaz}
\seExplica{??}{??}
\usa{Bool}

\begin{operaciones}
\InterfazFuncion{Test2}
{foo,foo}{bar}
[\dots]
{\dots}
[\dots]
[\dots]
[\dots]
[\dots]\\

\InterfazFuncion{Test}
{foo,foo}{bar}
[\dots]
{\dots}
[\dots]
[\dots]
[\dots]
[\dots]
\end{operaciones}

\subsection{Implementación}

\subsubsection*{Representación}
\begin{Estructura}{foo}
\begin{Tupla}
  \tupItem{foo}{bar}
  \tupItem{foo}{bar}
\end{Tupla}
\end{Estructura}

\subsubsection*{Invariante de Representación}
\Rep{foo}

\par\vspace*{3ex}%

\subsubsection*{Función de Abstracción}
\Abs[estr]{foo}[e]{p}{bar}

\subsubsection*{Algoritmos}
\begin{algorithm}[H]
  \caption{\textsc{hacerGuia}(\textbf{in} \textit{A} : \texttt{guia}, \textit{parámetroInútil} : \texttt{Nat}) $\longrightarrow$ \texttt{bool}}
  \begin{algorithmic}[1]
    \State \asignar{i}{0} \Comment{esto es $\Theta$(1)}

    \State \asignar{n}{guia.\text{cantEjercicios()}} \Comment{$\bigO$(1)}

    \State \asignar{consultas}{\textsc{diccVacio}}

    \State \textsc{prepararMate()} \Comment{$\Omega$($n^n$)}

    \While {$i < n$}
    \State \textsc{pensarEjercicio(i)}
    \If{\textsc{tengoConsultas($i$)}}
      \State \textsc{escribirConsultasEjercicio($i$,$consultas$)}
    \Else
      \State \textsc{comerBizochito()}
    \EndIf
      \State \textsc{comerBizochito()}
    \EndWhile

    \For{miVariable}
         \State hacer algo
    \EndFor

    \State \Return{\textsc{vacio?}($consultas$)}
  \end{algorithmic}
\end{algorithm}
\begin{algorithm}[H]
  \caption{\textsc{hacerGuia}(\textbf{in} \textit{A} : \texttt{guia}, \textit{parámetroInútil} : \texttt{Nat}) $\longrightarrow$ \texttt{bool}}
  \begin{algorithmic}[1]
    \State \asignar{i}{0} \Comment{esto es $\Theta$(1)}

    \State \asignar{n}{guia.\text{cantEjercicios()}} \Comment{$\bigO$(1)}

    \State \asignar{consultas}{\textsc{diccVacio}}

    \State \textsc{prepararMate()} \Comment{$\Omega$($n^n$)}

    \While {$i < n$}
    \State \textsc{pensarEjercicio(i)}
    \If{\textsc{tengoConsultas($i$)}}
      \State \textsc{escribirConsultasEjercicio($i$,$consultas$)}
    \Else
      \State \textsc{comerBizochito()}
    \EndIf
      \State \textsc{comerBizochito()}
    \EndWhile

    \For{miVariable}
         \State hacer algo
    \EndFor

    \State \Return{\textsc{vacio?}($consultas$)}
  \end{algorithmic}
\end{algorithm}
\newpage

\section{Módulo Juego}
\subsection{Interfaz}
\subsection{Implementación}
\newpage

\section{Módulos auxiliares}
\subsection{Módulo Foo}
\subsubsection{Interfaz}
\subsubsection{Implementación}




%%% Local Variables:
%%% mode: latex
%%% TeX-master: "main"
%%% End:
