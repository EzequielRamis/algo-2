% !TEX root = ../main.tex

%%%%%%%%%%%%%%%%%%%%%%%%%%%%%%%%%%%%%%%%%%%%%%%%%%%%%%%%%%%%
% COMANDOS QUE ALGUN DIA PUEDAN FORMAR UN PAQUETE.
%%%%%%%%%%%%%%%%%%%%%%%%%%%%%%%%%%%%%%%%%%%%%%%%%%%%%%%%%%%%
\newcommand{\moduloNombre}[1]{\textbf{#1}}

\let\NombreFuncion=\textsc
\let\TipoVariable=\texttt
\let\ModificadorArgumento=\textbf
\newcommand{\res}{$res$\xspace}
\newcommand{\tab}{\hspace*{7mm}}

\newcommandx{\TipoFuncion}[3]{%
  \NombreFuncion{#1}(#2) \ifx#3\empty\else $\to$ \res\,: \TipoVariable{#3}\fi%
}
\newcommand{\In}[2]{\ModificadorArgumento{in} \ensuremath{#1}\,: \TipoVariable{#2}\xspace}
\newcommand{\Out}[2]{\ModificadorArgumento{out} \ensuremath{#1}\,: \TipoVariable{#2}\xspace}
\newcommand{\Inout}[2]{\ModificadorArgumento{in/out} \ensuremath{#1}\,: \TipoVariable{#2}\xspace}
\newcommand{\Aplicar}[2]{\NombreFuncion{#1}(#2)}

\newlength{\IntFuncionLengthA}
\newlength{\IntFuncionLengthB}
\newlength{\IntFuncionLengthC}
%InterfazFuncion(nombre, argumentos, valor retorno, precondicion, postcondicion, complejidad, descripcion, aliasing)
\newcommandx{\InterfazFuncion}[9][4=true,6,7,8,9]{%
  \hangindent=\parindent
  \TipoFuncion{#1}{#2}{#3}\\%
  \textbf{Pre} $\equiv$ \{#4\}\\%
  \textbf{Post} $\equiv$ \{#5\}%
  \ifx#6\empty\else\\\textbf{Complejidad:} #6\fi%
  \ifx#7\empty\else\\\textbf{Descripción:} #7\fi%
  \ifx#8\empty\else\\\textbf{Aliasing:} #8\fi%
  \ifx#9\empty\else\\\textbf{Requiere:} #9\fi%
}

\newenvironment{Interfaz}{%
  \parskip=2ex%
  \noindent\textbf{\Large Interfaz}%
  \par%
}{}

\newenvironment{Representacion}{%
  \vspace*{2ex}%
  \noindent\textbf{\Large Representación}%
  \vspace*{2ex}%
}{}

\newenvironment{Algoritmos}{%
  \vspace*{2ex}%
  \noindent\textbf{\Large Algoritmos}%
  \vspace*{2ex}%
}{}


\newcommand{\Titulo}[1]{
  \vspace*{1ex}\par\noindent\textbf{\large #1}\par
}

\newenvironmentx{Estructura}[2][2={estr}]{%
  \par\vspace*{2ex}%
  \TipoVariable{#1} \textbf{se representa con} \TipoVariable{#2}%
  \par\vspace*{1ex}%
}{%
  \par\vspace*{2ex}%
}%

\newboolean{EstructuraHayItems}
\newlength{\lenTupla}
\newenvironmentx{Tupla}[1][1={estr}]{%
    \settowidth{\lenTupla}{\hspace*{3mm}donde \TipoVariable{#1} es \TipoVariable{tupla}$($}%
    \addtolength{\lenTupla}{\parindent}%
    \hspace*{3mm}donde \TipoVariable{#1} es \TipoVariable{tupla}$($%
    \begin{minipage}[t]{\linewidth}%
    \setboolean{EstructuraHayItems}{false}%
}{%
    $)$%
    \end{minipage}
}

\newcommandx{\tupItem}[3][1={\ }]{%
    %\hspace*{3mm}%
    \ifthenelse{\boolean{EstructuraHayItems}}{%
        , #1%
    }{}%
    \emph{#2}: \TipoVariable{#3}%
    \setboolean{EstructuraHayItems}{true}%
}

\newcommandx{\tupItemG}[3][1={\ }]{%
    %\hspace*{3mm}%
    \ifthenelse{\boolean{EstructuraHayItems}}{%
        , \\ #1%
    }{}%
    \emph{#2}: \TipoVariable{#3}%
    \setboolean{EstructuraHayItems}{true}%
}

\newcommandx{\RepFc}[3][1={estr},2={e}]{%
  \tadOperacion{Rep}{#1}{bool}{}%
  \tadAxioma{Rep($#2$)}{#3}%
}%

\newcommandx{\Rep}[3][1={estr},2={e}]{%
  \tadOperacion{Rep}{#1}{bool}{}%
  \tadAxioma{Rep($#2$)}{true \ssi #3}%
}%

\newcommandx{\Abs}[5][1={estr},3={e}]{%
  \tadOperacion{Abs}{#1/#3}{#2}{Rep($#3$)}%
  \settominwidth{\hangindent}{Abs($#3$) \igobs #4: #2 $\mid$ }%
  \addtolength{\hangindent}{\parindent}%
  Abs($#3$) \igobs #4: #2 $\mid$ #5%
}%

\newcommandx{\AbsFc}[4][1={estr},3={e}]{%
  \tadOperacion{Abs}{#1/#3}{#2}{Rep($#3$)}%
  \tadAxioma{Abs($#3$)}{#4}%
}%


\newcommand{\DRef}{\ensuremath{\rightarrow}}
