\section{Módulos auxiliares}
\subsection{Módulo Variante}
\begin{interfaz}{\subsubsection{Interfaz}}
  \seExplica{Variante}{variante}\\
  \usa{\falta}
  \par\noindent
  \begin{operaciones}
    \InterfazFuncion{nuevaVariante}
    {
      \begin{adjustwidth}{2em}{}
      \In{n}{nat},\\
      \In{f}{nat},\\
      \In{puntajes}{dicc(letra, nat)},\\
      \In{legítimas}{conj(secu(letra))}
      \end{adjustwidth}
    }{variante}
    [$n>0\land f>0$]
    {$res \igobs nuevaVariante(n,f,puntajes,leg\acute{\imath}timas)$}
    [$O(1)$]
    [Genera una variante de juego.]
    [\falta]\\

    \noindent\InterfazFuncion{tamañoTablero}
    {\In{v}{variante}}{nat}
    [true]
    {$res \igobs tama\tilde{n}oTablero(v)$}
    [$O(1)$]
    [Devuelve el tamaño del tablero.]
    [\falta]\\

    \noindent\InterfazFuncion{fichasPorJugador}
    {\In{v}{variante}}{nat}
    [true]
    {$res \igobs \#fichas(v)$}
    [$O(1)$]
    [Devuelve la cantidad de fichas que debe de tener cada jugador.]
    [\falta]\\

    \noindent\InterfazFuncion{puntajeLetra}
    {\In{v}{variante}, \In{l}{letra}}{nat}
    [true]
    {$res \igobs \#fichas(v)$}
    [$O(1)$]
    [Devuelve la cantidad de fichas que debe de tener cada jugador.]
    [\falta]\\

  \end{operaciones}
\end{interfaz}

\subsection{Módulo Ocurrencia}
\subsection{Módulo Notificación}

%%% Local Variables:
%%% mode: latex
%%% TeX-master: "main"
%%% End:
