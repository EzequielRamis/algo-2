\section{Módulos auxiliares}
\subsection{Módulo Letra}
Se asume una implementación acorde\footnote{Una buena opción es usar un \href{https://en.wikipedia.org/wiki/Enumerated_type}{Enumerado}.} al módulo de género \texttt{letra} con las siguientes operaciones en la interfaz (todas con órden de complejidad $O(1)$):
\begin{itemize}
    \item $\textsc{dom}\quad :\quad\quad\quad \rightarrow\texttt{nat}$ \---- Tamaño del dominio del tipo \texttt{letra}. Corresponde con la variable $A$ de su especificación.
    \item $\textsc{ord}\quad:\texttt{letra}\rightarrow\texttt{nat}$ \---- Dada una letra, devuelve su correspondiente índice.
    \item $\textsc{ord}^{-1}: \texttt{nat}\ n\rightarrow\texttt{letra}\ \{n < A\}$ \---- Dado un índice, devuelve su correspondiente letra.
\end{itemize}

\subsection{Módulo Variante (Trie)}
\begin{interfaz}{\subsubsection{Interfaz}}
  \seExplica{Variante}{variante}\\
  \usa{\falta}
  \par\noindent
  \begin{operaciones}
    \InterfazFuncion{nuevaVariante}
    {
      \begin{adjustwidth}{2em}{}
      \In{n}{nat},\\
      \In{f}{nat},\\
      \In{puntajes}{dicc(letra, nat)},\\
      \In{legítimas}{conj(secu(letra))}
      \end{adjustwidth}
    }{variante}
    [$n>0\land f>0$]
    {$res\igobs nuevaVariante(n,f,puntajes,leg\acute{\imath}timas)$}
    [$O(\#leg\acute{\imath}timas\cdot\Lmax)$]
    [Genera una variante de juego.]
    [\falta]\\

    \noindent\InterfazFuncion{tamañoTablero}
    {\In{v}{variante}}{nat}
    [true]
    {$res\igobs tama\tilde{n}oTablero(v)$}
    [$O(1)$]
    [Devuelve el tamaño del tablero.]
    [\falta]\\

    \noindent\InterfazFuncion{fichasPorJugador}
    {\In{v}{variante}}{nat}
    [true]
    {$res\igobs \#fichas(v)$}
    [$O(1)$]
    [Devuelve la cantidad de fichas que debe de tener cada jugador.]
    [\falta]\\

    \noindent\InterfazFuncion{puntajeLetra}
    {\In{v}{variante}, \In{l}{letra}}{nat}
    [true]
    {$res\igobs puntajeLetra(v,l)$}
    [$O(1)$]
    [Devuelve el puntaje de una letra.]
    [\falta]\\

    \noindent\InterfazFuncion{palabraLegítima?}
    {\In{v}{variante}, \In{l}{secu(letra)}}{bool}
    [true]
    {$res\igobs palabraLeg\acute{\imath}tima(v,l)$}
    [$O(\Lmax)$]
    [Determina si una palabra es legítima dentro de la variante de juego.]
    [\falta]\\

    \noindent\InterfazFuncion{longPalabraMásLarga}
    {\In{v}{variante}}{nat}
    [true]
    {
      \begin{adjustwidth}{2em}{}

      $(\exists p:\texttt{secu(letra)})(res\igobs long(p)\land palabraLeg\acute{\imath}tima?(v,p)\land\\(\forall p_{2}:\texttt{secu(letra)})( palabraLeg\acute{\imath}tima?(v,p_{2})\implies res\geq long(p_{2})))$
      \end{adjustwidth}
    }
    [$O(1)$]
    [Obtiene la longitud de la palabra legítima más larga de la variante.]
    [\falta]
  \end{operaciones}
\end{interfaz}
\subsubsection{Implementación}

\subsection{Módulo Tablero}
\begin{interfaz}{\subsubsection{Interfaz}}
  \seExplica{Tablero}{tab}\\
  \usa{\falta}
  \par\noindent
  \begin{operaciones}
    \InterfazFuncion{nuevoTablero}
    {
      \In{n}{nat}
    }{tab}
    [$n>0$]
    {$res\igobs nuevoTablero(n)$}
    [$O(N^{2})$]
    [Genera un tablero de tamaño $n$.]
    [\falta]\\

    \noindent\InterfazFuncion{ponerLetra}
    {
      \In{t}{tab},
      \In{i}{nat},
      \In{j}{nat},
      \In{l}{letra},
      \In{tm}{nat}
    }{tab}
    [$enTablero(t,i,j)\yluego\lnot hayLetra?(t,i,j)$]
    {$res\igobs nuevoTablero(n)$}
    [$O(N^{2})$]
    [Genera un tablero de tamaño $n$.]

    [\falta]\\
    \noindent\InterfazFuncion{tamañoTablero}
    {\In{v}{variante}}{nat}
    [true]
    {$res\igobs tama\tilde{n}oTablero(v)$}
    [$O(1)$]
    [Devuelve el tamaño del tablero.]
    [\falta]\\

    \noindent\InterfazFuncion{fichasPorJugador}
    {\In{v}{variante}}{nat}
    [true]
    {$res\igobs \#fichas(v)$}
    [$O(1)$]
    [Devuelve la cantidad de fichas que debe de tener cada jugador.]
    [\falta]\\

    \noindent\InterfazFuncion{puntajeLetra}
    {\In{v}{variante}, \In{l}{letra}}{nat}
    [true]
    {$res\igobs puntajeLetra(v,l)$}
    [$O(1)$]
    [Devuelve el puntaje de una letra.]
    [\falta]\\

    \noindent\InterfazFuncion{palabraLegítima?}
    {\In{v}{variante}, \In{l}{secu(letra)}}{bool}
    [true]
    {$res\igobs palabraLeg\acute{\imath}tima(v,l)$}
    [$O(\Lmax)$]
    [Determina si una palabra es legítima dentro de la variante de juego.]
    [\falta]\\

    \noindent\InterfazFuncion{longPalabraMásLarga}
    {\In{v}{variante}}{nat}
    [true]
    {
      \begin{adjustwidth}{2em}{}

      $(\exists p:\texttt{secu(letra)})(res\igobs long(p)\land palabraLeg\acute{\imath}tima?(v,p)\land\\(\forall p_{2}:\texttt{secu(letra)})( palabraLeg\acute{\imath}tima?(v,p_{2})\implies res\geq long(p_{2})))$
      \end{adjustwidth}
    }
    [$O(1)$]
    [Obtiene la longitud de la palabra legítima más larga de la variante.]
    [\falta]
  \end{operaciones}
\end{interfaz}
\subsubsection{Implementación}

\subsection{Módulo Ocurrencia}
es renombre de conj(tupla(nat,nat,letra)) es fichas del jugador a mano, no incluimos las que están alineadas en el tablero

\subsection{Módulo Notificación}
asumimos que existe el tipo notif

%%% Local Variables:
%%% mode: latex
%%% TeX-master: "main"
%%% End:
