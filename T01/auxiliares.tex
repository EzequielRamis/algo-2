\section{Módulos auxiliares}
\subsection{Módulo Letra}
Se asume una implementación acorde\footnote{Una buena opción es usar un \href{https://en.wikipedia.org/wiki/Enumerated_type}{Enumerado}.} al módulo de género \texttt{letra} con las siguientes operaciones en la interfaz (todas con órden de complejidad $O(1)$):
\begin{itemize}
    \item $\textsc{dom}\quad :\quad\quad\quad \rightarrow\texttt{nat}$ \---- Tamaño del dominio del tipo \texttt{letra}. Corresponde con la variable $A$ de su especificación.
    \item $\textsc{ord}\quad:\texttt{letra}\rightarrow\texttt{nat}$ \---- Dada una letra, devuelve su correspondiente índice.
    \item $\textsc{ord}^{-1}: \texttt{nat}\ n\rightarrow\texttt{letra}\ \{n < A\}$ \---- Dado un índice, devuelve su correspondiente letra.
\end{itemize}

\subsection{Módulo Variante}
\begin{interfaz}{\subsubsection{Interfaz}}
  \seExplica{Variante}{variante}\\
  \usa{Diccionario Lineal, Conjunto Lineal, Lista, Letra, ¿Trie de Palabras?}
  \par\noindent
  \begin{operaciones}
    \InterfazFuncion{nuevaVariante}
    {
      \begin{adjustwidth}{2em}{}
      \In{n}{nat},\\
      \In{f}{nat},\\
      \In{puntajes}{dicc(letra, nat)},\\
      \In{legítimas}{conj(lista(letra))}
      \end{adjustwidth}
    }{variante}
    [$n>0\land f>0$]
    {$res\igobs nuevaVariante(n,f,puntajes,leg\acute{\imath}timas)$}
    [$O(|\Sigma|+\#leg\acute{\imath}timas\cdot\Lmax)$]
    [Genera una variante de juego.]
    [Se pasa $puntajes:\texttt{dicc(letra,nat)}$ como referencia \textbf{no} modificable.\\
    \-\hspace{4.7em}Se pasa $leg\acute\imath timas:\texttt{conj(lista(letra))}$ como referencia \textbf{no} modificable.]\\

    \noindent\InterfazFuncion{tamañoTablero}
    {\In{v}{variante}}{nat}
    [true]
    {$res\igobs tama\tilde{n}oTablero(v)$}
    [$O(1)$]
    [Devuelve el tamaño del tablero.]\\

    \noindent\InterfazFuncion{fichasPorJugador}
    {\In{v}{variante}}{nat}
    [true]
    {$res\igobs \#fichas(v)$}
    [$O(1)$]
    [Devuelve la cantidad de fichas que debe de tener cada jugador.]\\

    \noindent\InterfazFuncion{puntajeLetra}
    {\In{v}{variante}, \In{l}{letra}}{nat}
    [true]
    {$res\igobs puntajeLetra(v,l)$}
    [$O(1)$]
    [Devuelve el puntaje de una letra.]\\

    \noindent\InterfazFuncion{palabraLegítima?}
    {\In{v}{variante}, \In{p}{lista(letra)}}{bool}
    [true]
    {$res\igobs palabraLeg\acute{\imath}tima(v,p)$}
    [$O(\Lmax)$]
    [Determina si una palabra es legítima dentro de la variante de juego.]
    [Se pasa $p:\texttt{lista(letra)}$ como referencia \textbf{no} modificable.]\\

    \noindent\InterfazFuncion{longPalabraMásLarga}
    {\In{v}{variante}}{nat}
    [true]
    {
      \begin{adjustwidth}{2em}{}

      $(\exists p:\texttt{secu(letra)})(res\igobs long(p)\land palabraLeg\acute{\imath}tima?(v,p)\land\\(\forall p_{2}:\texttt{secu(letra)})( palabraLeg\acute{\imath}tima?(v,p_{2})\implies res\geq long(p_{2})))$
      \end{adjustwidth}
    }
    [$O(1)$]
    [Obtiene la longitud de la palabra legítima más larga de la variante.]
  \end{operaciones}
\end{interfaz}
\subsubsection{Implementación}
\subsubsection*{Representación}
\begin{Estructura}{variante}[variante\_estr]
\begin{Tupla}[variante\_estr]
\begin{adjustwidth}{3em}{}\ \
  \tupItem{tablero}{nat}\\
  \tupItem{fichas}{nat}\\
  \tupItem{puntaje}{array\_dimensionable(nat)}\\
  \tupItem{palabras}{triePalabra}\\
  \tupItem{\#palabraMásLarga}{triePalabra}
\end{adjustwidth}
\end{Tupla}
\end{Estructura}

\subsubsection*{Invariante de Representación}
\Rep[variante\_estr]{$
  e.tablero>0\land e.fichas>0\land\\
  tam(e.puntaje)=\textsc{dom}()\yluego(\forall l:\texttt{letra})(e.puntaje[\textsc{ord}(l)]>0)\land\\
  \lnot(\exists p:\texttt{lista(letra)})(\textsc{Definida?}(e.palabras,p)\implies \textsc{Longitud}(p)>e.\#palabraM\acute{a}sLarga)
$}

\subsubsection*{Función de Abstracción}
\Abs[variante\_estr]{variante}[e]{V}{$
  e.tablero\igobs tama\tilde{n}oTablero(V)\land\\
  e.fichas\igobs \#fichas(V)\land\\
  (\forall l:\texttt{letra})(e.puntaje[\textsc{ord}(l)]\igobs puntajeLetra(V,l))\land\\
  (\forall p:\texttt{secu(letra)})(p\in e.palabras\igobs palabraLeg\acute{\imath}tima?(V,l))
$}

\subsubsection{Algoritmos}
\begin{algorithm}[H]
  \defun{iNuevaVariante}{\In{n}{nat}, \In{f}{nat}, \In{puntajes}{dicc(letra,nat)}, \In{legítimas}{conj(secu(letra))}}{variante\_estr}
  \begin{algorithmic}[1]
    \State\asignar{res.tablero}{n}
    \State\asignar{res.fichas}{f}
    \State\asignar{res.puntaje}{\textsc{CrearArreglo}(\textsc{dom}())}
    \For{$0\leq i<tam(res.puntaje)$}\Comment$O(|\Sigma|)$
      \If{$\textsc{Definido?}(puntajes,\textsc{ord}^{-1}(i))$}
        \State\asignar{res.puntaje[i]}{\textsc{Significado}(puntajes,\textsc{ord}^{-1}(i))}
      \Else
        \State\asignar{res.puntaje[i]}{1}
      \EndIf
    \EndFor
    \State\asignar{res.palabras}{\textsc{Vacía}()}
    \State\asignar{res.\#palabraM\acute{a}sLarga}{0}
    \For{$ $\asignar{lgIt}{\textsc{CrearIt}(lg)}$;\ \textsc{HaySiguiente}(lgIt);\ \textsc{Avanzar}(lgIt)$}\Comment{$O(\#leg\acute{\imath}timas)$}
      \State\asignar{palabra}{\textsc{Siguiente}(lgIt)}
      \State$\textsc{Definir}(res.palabras,palabra)$\Comment$O(\Lmax)$
      \If{$\textsc{Longitud}(palabra)>res.\#palabraM\acute{a}sLarga$}\Comment$O(1)$
        \State\asignar{res.\#palabraM\acute{a}sLarga}{\textsc{Longitud}(palabra)}
      \EndIf
    \EndFor
    \State\Return$res$
  \end{algorithmic}
\end{algorithm}
\begin{algorithm}[H]
  \defun{iTamañoTablero}{\In{v}{variante\_estr}}{nat}
  \begin{algorithmic}[1]
    \State\Return$v.tablero$
  \end{algorithmic}
\end{algorithm}
\begin{algorithm}[H]
  \defun{iFichasPorJugador}{\In{v}{variante\_estr}}{nat}
  \begin{algorithmic}[1]
    \State\Return$v.fichas$
  \end{algorithmic}
\end{algorithm}
\begin{algorithm}[H]
  \defun{iPuntajeLetra}{\In{v}{variante\_estr}, \In{l}{letra}}{nat}
  \begin{algorithmic}[1]
    \State\Return$v.puntaje[\textsc{ord}(l)]$
  \end{algorithmic}
\end{algorithm}
\begin{algorithm}[H]
  \defun{iPalabraLegítima?}{\In{v}{variante\_estr}, \In{p}{secu(letra)}}{bool}
  \begin{algorithmic}[1]
    \State\Return$\textsc{Definida?}(v.palabras,p)$\Comment$O(\Lmax)$
  \end{algorithmic}
\end{algorithm}
\begin{algorithm}[H]
  \defun{iLongPalabraMásLarga}{\In{v}{variante\_estr}}{nat}
  \begin{algorithmic}[1]
    \State\Return$v.\#palabraM\acute{a}sLarga$
  \end{algorithmic}
\end{algorithm}

\subsection{Módulo Ocurrencia}
Es renombre de \texttt{conj(tupla(nat,nat,letra))} con las siguientes operaciones auxiliares.\\
\begin{adjustwidth}{3em}{0em}
\begin{operaciones}
  \InterfazFuncion{esHorizontal?}
  {
    \In{o}{ocurrencia}
  }{bool}
  [true]
  {$res\igobs true\iff(\forall f,f':\texttt{tupla(nat,nat,letra)})(f,f'\in o\land f\neq f'\implies \pi_{2}(f)=\pi_{2}(f'))$}
  [$O(\#o^{2})$]
  [Determina si una ocurrencia está alineada horizontalmente.]\\

  \noindent\InterfazFuncion{esVertical?}
  {
    \In{o}{ocurrencia}
  }{bool}
  [true]
  {$res\igobs true\iff(\forall f,f':\texttt{tupla(nat,nat,letra)})(f,f'\in o\land f\neq f'\implies \pi_{1}(f)=\pi_{1}(f'))$}
  [$O(\#o^{2})$]
  [Determina si una ocurrencia está alineada verticalmente.]\\

  \noindent\InterfazFuncion{haySuperpuestas?}
  {
    \In{o}{ocurrencia}
  }{bool}
  [true]
  {$res\igobs false\iff
      (\forall p,q:\texttt{nat})(\forall l,l':\texttt{letra})(
      \{\langle p,q,l\rangle,\langle p,q,l'\rangle\}\subseteq o\implies l=l')$}
  [$O(\#o^{2})$]
  [Determina si existen fichas distintas en una misma coordenada.]
\end{operaciones}
\end{adjustwidth}

\subsubsection{Algoritmos auxiliares}
\begin{algorithm}[H]
  \defun{iEsHorizontal?}{\In{o}{ocurrencia}}{bool}
  \begin{algorithmic}[1]
    \State\asignar{colns}{\textsc{Vacío}()}\Comment{Conjunto Lineal}
    \For{$ $\asignar{oIt}{\textsc{CrearIt}(o)}$;\ \textsc{HaySiguiente}(oIt);\ \textsc{Avanzar}(oIt)$}\Comment{$O(\#o)$}
      \State\asignar{ficha}{\textsc{Siguiente}(oIt)}
      \State$\textsc{Agregar}(colns,\pi_{2}(ficha)))$\Comment{$O(\#o)$}
    \EndFor
    \State\Return{\textsc{Cardinal}(colns)=1}
  \end{algorithmic}
\end{algorithm}

\begin{algorithm}[H]
  \defun{iEsVertical?}{\In{o}{ocurrencia}}{bool}
  \begin{algorithmic}[1]
    \State\asignar{filas}{\textsc{Vacío}()}\Comment{Conjunto Lineal}
    \For{$ $\asignar{oIt}{\textsc{CrearIt}(o)}$;\ \textsc{HaySiguiente}(oIt);\ \textsc{Avanzar}(oIt)$}\Comment{$O(\#o)$}
      \State\asignar{ficha}{\textsc{Siguiente}(oIt)}
      \State$\textsc{Agregar}(filas,\pi_{1}(ficha)))$\Comment{$O(\#o)$}
    \EndFor
    \State\Return{\textsc{Cardinal}(filas)=1}
  \end{algorithmic}
\end{algorithm}

\begin{algorithm}[H]
  \defun{iHaySuperpuestas?}{\In{o}{ocurrencia}}{bool}
  \begin{algorithmic}[1]
    \For{$ $\asignar{oIt}{\textsc{CrearIt}(o)}$;\ \textsc{HaySiguiente}(oIt);\ \textsc{Avanzar}(oIt)$}\Comment{$O(\#o)$}
        \State\asignar{ficha}{\textsc{Siguiente}(oIt)}
        \For{$ $\asignar{oItSig}{\textsc{Avanzar}(copy(oIt))}$;\ \textsc{HaySiguiente}(oItSig);\ \textsc{Avanzar}(oItSig)$}\Comment{$O(\#o)$}
          \State\asignar{ficha'}{\textsc{Siguiente}(oItSig)}
          \If{$\pi_{1}(ficha)=\pi_{1}(fichaSig)\land\pi_{2}(ficha)=\pi_{2}(fichaSig)$}
          \State\Return true
          \EndIf
        \EndFor
    \EndFor
    \State\Return false
  \end{algorithmic}
\end{algorithm}

\subsection{Módulo Notificación}
Asumimos que existe el tipo \texttt{notif}.

\subsection{Módulo Trie de Palabras}
\begin{interfaz}{\subsubsection{Interfaz}}
  \seExplica{Conjunto(Secuencia(Letra))}{triePalabra}\\
  \par\noindent
  \begin{operaciones}
    \InterfazFuncion{Vacío}
    {}{triePalabra}
    [true]
    {$res\igobs\emptyset$}
    [$O(1)$]
    [Genera un trie de palabras.]\\

    \noindent\InterfazFuncion{Definir}
    {\Inout{t}{triePalabra}, \In{p}{lista(letra)}}{}
    [$p\notin t$]
    {$p\in t$}
    [$O(\Lmax)$]
    [Define una palabra en el trie.]
    [Se pasa $p:\texttt{lista(letra)}$ como referencia \textbf{no} modificable.]\\

    \noindent\InterfazFuncion{Definida?}
    {\In{t}{triePalabra}, \In{p}{lista(letra)}}{bool}
    [true]
    {$res\igobs p\in t$}
    [$O(\Lmax)$]
    [Determina si una palabra está definida en el trie.]
    [Se pasa $p:\texttt{lista(letra)}$ como referencia \textbf{no} modificable.]
  \end{operaciones}
\end{interfaz}
\subsubsection{Implementación}
\subsubsection*{Representación}
\begin{Estructura}{triePalabra}[trie\_estr]
\begin{Tupla}[trie\_estr]
\begin{adjustwidth}{3em}{}\ \
  \tupItem{hijos}{array\_dimensionable(puntero(trie\_estr))}\\
  \tupItem{fin?}{bool}
\end{adjustwidth}
\end{Tupla}
\end{Estructura}

\subsubsection*{Invariante de Representación}
\Rep[trie\_estr]{$
  tam(e.hijos)=\textsc{dom}()\yluego\\
  (\forall h:\texttt{nat})(h<tam(e.hijos)\impluego definido?(e.hijos[h])\land\\
  (\forall h:\texttt{nat})(h<tam(e.hijos)\impluego e.hijos[h]=\textsc{NULL})\implies e.fin?=true
$}

\subsubsection*{Función de Abstracción}
\Abs[trie\_estr]{conj(secu(letra))}[t]{C}{$
  <>\in C\implies C\igobs palabrasDeTrie(t)\oluego\\
  C\igobs palabrasDeTrie(t)-\{<>\}
$}

\begin{adjustwidth}{12em}{}$
  \textbf{donde}\\
  % tama\tilde{n}oTrie: \texttt{trie\_estr}\longrightarrow\texttt{nat}\\
  % tama\tilde{n}oTrie(t)\equiv sumarHijosDesde(t,0)-1\\\\
  % sumarHijosDesde: \texttt{trie\_estr}\times\texttt{nat}\longrightarrow\texttt{nat}\\
  % sumarHijosDesde(t,k)\equiv$\IFL{1em}G{k=\textsc{dom}()}THEN{\LIF\ t.fin?\ \LTHEN\ 1\ \LELSE\ 0\ \LFI}ELSE{sumarHijo(t.hijos[k])+sumarHijosDesde(t,k+1)}FI$\\
  % sumarHijo: \texttt{puntero(trie\_estr)}\longrightarrow\texttt{nat}\\
  % sumarHijo(p)\equiv$\IFL{1em}G{p=NULL}THEN{0}ELSE{sumarHijosDesde(*p,0)}FI$\\\\
  palabrasDeTrie: \texttt{trie\_estr}\longrightarrow\texttt{conj(secu(letra))}\\
  palabrasDeTrie(t)\equiv formarPalabrasDesde(t,0,<>)\\\\
  formarPalabrasDesde: \texttt{trie\_estr}\times\texttt{nat}\times\texttt{secu(letra)}\longrightarrow\texttt{conj(secu(letra))}\\
  formarPalabrasDesde(t,k,pre)\equiv$\IFL{1em}G{k=\textsc{dom}()}THEN{\LIF\ t.fin?\ \LTHEN\ Ag(pre,\emptyset)\ \LELSE\ \emptyset\ \LFI}ELSE{formarPalabras(t.hijos[k],pre\circulito\textsc{ord}^{-1}(k))\\\cup formarPalabrasDesde(t,k+1,pre)}FI$\\
  formarPalabras: \texttt{puntero(trie\_estr)}\times\texttt{secu(letra)}\longrightarrow\texttt{conj(secu(letra))}\\
  formarPalabras(p,pre)\equiv$\IFL{1em}G{p=NULL}THEN{\emptyset}ELSE{formarPalabrasDesde(*p,0,pre)}FI$
$\end{adjustwidth}

\subsubsection*{Algoritmos}
\begin{algorithm}[H]
  \defun{iVacío}{}{trie\_estr}
  \begin{algorithmic}[1]
    \State\asignar{res.fin?}{false}
    \State\asignar{res.hijos}{\textsc{CrearArreglo}(\textsc{dom}())}
    \For{$0\leq i<tam(res.hijos)$}\Comment$O(|\Sigma|)$
        \State\asignar{res.hijos[i]}{\textsc{NULL}}
    \EndFor
    \State\Return$res$
  \end{algorithmic}
\end{algorithm}
\begin{algorithm}[H]
  \defunv{iDefinir}{\Inout{t}{trie\_estr}, \In{p}{lista(letra)}}
  \begin{algorithmic}[1]
    \iffalse
      if !pit.haySig
        t.fin <- true
      else
        nodo <- t[ord(pit.sig)]
        while nodo != null
          letra <- pit.sig
          nodo <- (*nodo)[ord(letra)]
        while pit.haySig
          nodo <- &vacio
          letra <- pit.sig
          nodo <- (*nodo)[ord(letra)]
        (*nodo).fin <- true
    \fi
    \State\asignar{pIt}{\textsc{CrearIt}(p)}
    \If{$\lnot\textsc{HaySiguiente}(pIt)$}
      \State\asignar{t.fin?}{true}
    \Else
      \State\asignar{nodo}{t.hijos[\textsc{ord}(\textsc{Siguiente}(pIt))]}
      \While{$nodo\neq\textsc{NULL}$}
        \State\asignar{letra}{\textsc{Siguiente}(pIt)}
        \State\asignar{nodo}{(*nodo).hijos[\textsc{ord}(letra)]}
      \EndWhile
      \While{\textsc{HaySiguiente}(pIt)}
        \State\asignar{nodo}{\&\textsc{Vacío}()}
        \State\asignar{letra}{\textsc{Siguiente}(pIt)}
        \State\asignar{nodo}{(*nodo).hijos[\textsc{ord}(letra)]}
      \EndWhile
      \State\asignar{(*nodo).fin?}{true}
    \EndIf
  \end{algorithmic}
\end{algorithm}
\begin{algorithm}[H]
  \defun{iDefinida?}{\In{t}{trie\_estr}, \In{p}{lista(letra)}}{bool}
  \begin{algorithmic}[1]
    \iffalse
      if !pit.haySig
        return t.fin
      else
        nodo <- t[ord(pit.sig)]
        while nodo != null
          letra <- pit.sig
          nodo <- (*nodo)[ord(letra)]
        return !pit.haySig and (*nodo).fin
    \fi
    \State\asignar{pIt}{\textsc{CrearIt}(p)}
    \If{$\lnot\textsc{HaySiguiente}(pIt)$}
      \State\Return$t.fin?$
    \Else
      \State\asignar{nodo}{t.hijos[\textsc{ord}(\textsc{Siguiente}(pIt))]}
      \While{$nodo\neq\textsc{NULL}$}
        \State\asignar{letra}{\textsc{Siguiente}(pIt)}
        \State\asignar{nodo}{(*nodo).hijos[\textsc{ord}(letra)]}
      \EndWhile
      \State\Return$\lnot\textsc{HaySiguiente}(pIt)\land (*nodo).fin?$
    \EndIf
  \end{algorithmic}
\end{algorithm}

%%% Local Variables:
%%% mode: latex
%%% TeX-master: "main"
%%% End:
