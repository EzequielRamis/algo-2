\documentclass[10pt, a4paper]{article}
\usepackage[paper=a4paper, left=1.5cm, right=1.5cm, bottom=1.5cm, top=3.5cm]{geometry}
\usepackage[utf8]{inputenc}
% \usepackage[T1]{fontenc}
\usepackage[spanish]{babel}
\usepackage{indentfirst}
\usepackage{fancyhdr}
\usepackage{latexsym}
\usepackage{lastpage}
\usepackage{framed}
\usepackage{todonotes} % para dejar notitas de to-do!
\usepackage{aed2-symb,aed2-itef,aed2-tad,aed2-diseno}
\usepackage[colorlinks=true, linkcolor=black]{hyperref}
\usepackage{calc}
\usepackage{graphicx}
\usepackage{changepage}
%

% ========== Para escribir pseudo ==========
\usepackage{algorithm}
\usepackage[noend]{algpseudocode}  % "noend" es para no mostrar los endfor, endif
\algrenewcommand\alglinenumber[1]{\tiny #1:}  % Para que los numeros de linea del pseudo sean pequeños
\renewcommand{\thealgorithm}{}  % Que no aparezca el numero luego de "Algorithm"
\floatname{algorithm}{ }    % Entre {  } que quiero que aparezca en vez de "Algorithm"

% traducciones
% \algrenewcommand\algorithmicwhile{\textbf{mientras}}
% \algrenewcommand\algorithmicdo{\textbf{hacer}}
% \algrenewcommand\algorithmicreturn{\textbf{devolver}}
% \algrenewcommand\algorithmicif{\textbf{si}}
% \algrenewcommand\algorithmicthen{\textbf{entonces}}
% \algrenewcommand\algorithmicfor{\textbf{para}}
%=========================================================


\newcommand{\f}[1]{\text{#1}}
\renewcommand{\paratodo}[2]{$\forall~#2$: #1}
\newcommand{\numeroEjercicio}[1]{\textbf{\large{Ejercicio #1:}}\\}
\newcommand{\tituloSubEjercicio}[1]{$\newline$\tadNombre{#1:}}

\sloppy

\hypersetup{%
 % Para que el PDF se abra a página completa.
 pdfstartview= {FitH \hypercalcbp{\paperheight-\topmargin-1in-\headheight}},
 pdfauthor={Cátedra de Algoritmos y Estructuras de Datos II - DC - UBA},
 pdfkeywords={Template TADs básicos},
 pdfsubject={Template TADs básicos}
}

\parskip=5pt % 10pt es el tamaño de fuente

% Pongo en 0 la distancia extra entre ítemes.
\let\olditemize\itemize
\def\itemize{\olditemize\itemsep=0pt}

% Acomodo fancyhdr.
\pagestyle{fancy}
\thispagestyle{fancy}
\addtolength{\headheight}{1pt}
\lhead{Algoritmos y Estructuras de Datos II}
\rhead{$2^{\mathrm{do}}$ cuatrimestre de 2022}
\cfoot{\thepage /\pageref{LastPage}}
\renewcommand{\footrulewidth}{0.4pt}

\author{Algoritmos y Estructuras de Datos II, DC, UBA.}
\date{}
\title{Trabajo Práctico de Algoritmos y Estructuras de Datos II}

\NeedsTeXFormat{LaTeX2e}
\ProvidesPackage{caratula}[2003/4/13 v0.1 Para componer caratulas de TPs del DC]


% ----- Imprimir un mensajito al procesar un .tex que use este package -----

\typeout{Cargando package `caratula' v0.2 (21/4/2003)}


% ----- Algunas variables --------------------------------------------------

\let\Materia\relax
\let\Submateria\relax
\let\Titulo\relax
\let\Subtitulo\relax
\let\Grupo\relax


% ----- Comandos para que el usuario defina las variables ------------------

\def\materia#1{\def\Materia{#1}}
\def\submateria#1{\def\Submateria{#1}}
\def\titulo#1{\def\Titulo{#1}}
\def\subtitulo#1{\def\Subtitulo{#1}}
\def\grupo#1{\def\Grupo{#1}}


% ----- Token list para los integrantes ------------------------------------

\newtoks\intlist\intlist={}


% ----- Comando para que el usuario agregue integrantes

\def\integrante#1#2#3{\intlist=\expandafter{\the\intlist
    \rule{0pt}{1.2em}#1&#2&\tt #3\\[0.2em]}}


% ----- Macro para generar la tabla de integrantes -------------------------

\integrante{Guberman, Diego Andrés}{469/17}{diego98g@hotmail.com}
\integrante{Ramis Folberg, Ezequiel Leonel}{881/21}{ezequielramis.hello@gmail.com}
\integrante{Sabetay, Kevin Damian}{476/16}{kevin.sabetay96@gmail.com}

\def\tablaints{
    \begin{tabular}{|l@{\hspace{4ex}}c@{\hspace{4ex}}l|}
        \hline
        \rule{0pt}{1.2em}Integrante & LU & Correo electr\'onico\\[0.2em]
        \hline
        \the\intlist
        \hline
    \end{tabular}}

% ----- Macro para generar la parte reservada para la c�tedra -------------------------

\def\tablacatedra{%
    \\
    \textbf{Reservado para la c\'atedra}\par\bigskip
    \begin{tabular}{|c|c|c|}
        \hline
        \rule{0pt}{1.2em}Instancia & Docente & Nota\\[0.2em]
        \hline
        \rule{0pt}{1.2em}Primera entrega & \phantom{mmmmmmmmmmmmmmmmmm} & \phantom{mmmmmm} \\
        \hline
        \rule{0pt}{1.2em}Segunda entrega & & \\
        \hline
    \end{tabular}}

% ----- Codigo para manejo de errores --------------------------------------

\def\se{\let\ifsetuperror\iftrue}
\def\ifsetuperror{%
    \let\ifsetuperror\iffalse
    \ifx\Materia\relax\se\errhelp={Te olvidaste de proveer una \materia{}.}\fi
    \ifx\Titulo\relax\se\errhelp={Te olvidaste de proveer un \titulo{}.}\fi
    \edef\mlist{\the\intlist}\ifx\mlist\empty\se%
    \errhelp={Tenes que proveer al menos un \integrante{nombre}{lu}{email}.}\fi
    \expandafter\ifsetuperror}


% ----- Reemplazamos el comando \maketitle de LaTeX con el nuestro ---------

\def\maketitle{%
    \thispagestyle{empty}
    \begin{center}
    \vspace*{\stretch{2}}
    \materia{Algoritmos y Estructuras de Datos II}

    {\LARGE\textbf{\Materia}}\\[1em]
    \submateria{Trabajo Pr\'actico 1}
    \ifx\Submateria\relax\else{\Large \Submateria}\\[0.5em]\fi

%\def\titulo#1{\def\Titulo{#1}}
%\def\subtitulo#1{\def\Subtitulo{#1}}
%\def\grupo#1{\def\Grupo{#1}}
    \par\vspace{\stretch{1}}
    \begin{figure}[h!]
      \centering
      \includegraphics[width=0.5\linewidth]{caratula.png}
    \end{figure}
    \par\vspace{\stretch{1}}
    {\large Departamento de Computaci\'on}\\[0.5em]
    {\large Facultad de Ciencias Exactas y Naturales}\\[0.5em]
    {\large Universidad de Buenos Aires}
    \par\vspace{\stretch{3}}
    {\Large \textbf{\Titulo}}\\[0.8em]
    {\Large \Subtitulo}
    \par\vspace{\stretch{3}}
    \grupo
    \ifx\Grupo\relax\else\textbf{\Grupo}\par\bigskip\fi
    \tablaints
    \vspace*{\stretch{3}}
    \medskip
    \tablacatedra
    \end{center}
    \vspace*{\stretch{3}}
    \newpage
    }

% Comandos para cositas de complejidad

\newcommand{\bigO}{\mathcal{O}}
\newcommand{\Nat}{\mathbb{N}}
\newcommand{\R}{\mathbb{R}}
\newcommand{\Rpos}{\mathbb{R}_{>0}}
\newcommand{\eqdef}{\overset{\mathrm{def}}{=}}
\newcommand{\eqprop}{\overset{\mathrm{prop}}{=}}
%\newcommand{\ssi}{\leftrightarrow}

\newcommand{\seExplica}[2]{
\textbf{se explica con}: \NombreFuncion{#1}\par\noindent
\textbf{géneros}: \TipoVariable{#2}
}

\newcommand{\usa}[1] {
\textbf{usa}: \NombreFuncion{#1}\par
}

\newenvironment{interfaz}[1]
  {#1
  \adjustwidth{3em}{}}
  {\endadjustwidth}

\newenvironment{operaciones}
  {\textbf{operaciones}:
  \adjustwidth{2em}{}}
  {\endadjustwidth}

\newcommand{\asignar}[2]{$#1 \gets #2$}

\newcommand{\falta}{\todo[inline, inlinewidth=1.7em, noinlinepar]{??}}

\newcommand{\defun}[3]{
  \caption{\textsc{#1}(#2) $\longrightarrow$ \texttt{#3}}
}

\newcommand{\NULL}{\textsc{NULL}}

\begin{document}

\maketitle

\tableofcontents
\newpage

\section{Dudas}
\begin{itemize}
        \item Igualdad observacional sin tenerla definida en los TADs.
        \item Nuevo Juego: ¿El repo se pasa por parámetro o se genera ``aleatoriamente'' dentro de la función como un comportamiento automático? Si es el 1º caso se tendría que copiar para coincidir con la complejidad del ejercicio. Si es el segundo ver cómo se genera ``aleatoriamente''.
        \item ¿Qué tipo es \texttt{letra}? Si es \texttt{char} la complejidad de nuevoJuego no se cumple ($\Sigma$ sería muy grande). En caso de pasar por parámetro un conjunto de \texttt{chars} como alfabeto, habría que checkear que todas las fichas del repo inicial sean de ese alfabeto. Lo más probable es que sea un \texttt{enum}.
        \item ¿Se puede usar operaciones de interfaz dentro de \textbf{Pre} y \textbf{Post}? Por ejemplo para acceder a elemento de tablero, si es otro módulo.
        \item Si se quiere que \texttt{variante} sea un módulo, ¿qué tipo de parámetro de entrada se usan para la operación ``nuevaVariante''? ¿Las ya definidas en la representación? ¿Puede ser un módulo que no tenga interfaz y solamente el \textbf{Rep} y \textbf{Abs}?
        \item ¿Hay que poner el item ``usa'' o sólo completar los ``requiere''?
        \item Complejidad de operación \textsc{\#letraTieneJugador}.
\end{itemize}

\section{Preámbulo}
Antes de presentar los módulos, definimos las siguientes variables para las complejidades temporales:
\begin{itemize}
    \item $N$ \---- tamaño del tablero.
    \item $K$ \---- cantidad de jugadores.
    \item $|\Sigma|$ \---- cantidad de letras en el alfabeto.
    \item $F$ \---- cantidad de fichas por jugador.
    \item $L_{\texttt{máx}}$ \---- longitud de la palabra legítima más larga definida por la variante del juego de la que se trate.
\end{itemize}

%%% Local Variables:
%%% mode: latex
%%% TeX-master: "main"
%%% End:

\section{Módulo Juego}
\begin{interfaz}{\subsection{Interfaz}}
  \seExplica{Juego}{juego}\\\\
  \usa{Bool, Nat, Cola, Letra, Ocurrencia, Tablero, Variante}
  \par\noindent
  \begin{operaciones}
    \InterfazFuncion{nuevoJuego}
    {\In{k}{nat}, \In{v}{variante}, \In{r}{cola(letra)}}{juego}
    [$tama\tilde{n}o(r)\geq tama\tilde{n}oTablero(v)*tama\tilde{n}oTablero(v)+k*\#fichas(v)\land k>0$]
    {$res \igobs nuevoJuego(k,v,r)$}
    [$O(N^{2}+|\Sigma|K+FK)$]
    [Dada una cantidad de jugadores, una variante de juego y un repositorio de fichas, se inicia un nuevo juego con el tablero vacío.]
    [\falta]\\

    \noindent\InterfazFuncion{jugadaVálida?}
    {\In{j}{juego}, \In{o}{occurrencia}}{bool}
    [true]
    {$res\igobs jugadaV\acute{a}lida?(j,o)$}
    [$O(L_{\texttt{máx}}^{2})$]
    [Determina si una jugada es válida.]
    [\falta]\\

    \noindent\InterfazFuncion{ubicar}
    {\Inout{j}{juego}, \In{o}{occurrencia}}{}
    [$jugadaV\acute{a}lida(j,o) \land j\igobs J_{0} $]
    {$j\igobs ubicar(J_{0},o)$}
    [$O(m)$, donde $m$ es el número de fichas que se ubican.]
    [Ubica un conjunto de fichas en el juego.]
    [\falta]\\

    \noindent\InterfazFuncion{variante}
    {\In{j}{juego}}{variante}
    [true]
    {$res\igobs variante(j)$}
    [$O(1)$]
    [Obtiene información sobre la variante del juego.]
    [\falta]\\

    \noindent\InterfazFuncion{turno}
    {\In{j}{juego}}{nat}
    [true]
    {$res\igobs turno(j)$}
    [$O(1)$]
    [Obtiene al jugador del turno actual.]
    [\falta]\\

    \noindent\InterfazFuncion{puntaje}
    {\In{j}{juego}, \In{i}{nat}}{nat}
    [$i < \#jugadores(j)$]
    {$res\igobs puntaje(j,i)$}
    [$O(1+m\cdot L_{\texttt{máx}})$, donde $m$ es la cantidad de fichas que ubicó el
jugador desde la última vez que se invocó a esta operación.]
    [Obtiene el puntaje de un jugador.]
    [\falta]\\

    \noindent\InterfazFuncion{celdaOcupada?}
    {\In{J}{juego}, \In{i}{nat}, \In{j}{nat}}{bool}
    [$enTablero?(tablero(J),i,j)$]
    {$res\igobs hayLetra?(tablero(J),i,j)$}
    [$O(1)$]
    [Obtiene si el tablero en una coordenada $(i,j)$ está ocupado.]
    [\falta]\\

    \noindent\InterfazFuncion{celdaContenido}
    {\In{J}{juego}, \In{i}{nat}, \In{j}{nat}}{letra}
    [$enTablero?(tablero(J),i,j)\land_{L} hayLetra?(tablero(J),i,j)$]
    {$res\igobs letra(tablero(J),i,j)$}
    [$O(1)$]
    [Obtiene el contenido del tablero en una coordenada $(i,j)$ asumiendo que está ocupada.]
    [\falta]\\

    \noindent\InterfazFuncion{\#letraTieneJugador}
    {\In{j}{juego}, \In{x}{letra}, \In{i}{nat}}{nat}
    [$i < \#jugadores(j)$]
    {$res\igobs \#(x,fichas(j,i))$}
    [$O(1)$]
    [Dada una cierta letra $x$ del alfabeto, conocer cuántas fichas tiene un jugador de dicha letra.]
    [\falta]\\

  \end{operaciones}
\end{interfaz}

\subsection{Implementación}

\subsubsection*{Representación}
\begin{Estructura}{juego}[juego\_estr]
\begin{Tupla}[juego\_estr]
\begin{adjustwidth}{3em}{}\ \
  \tupItem{tablero}{array\_dimensionable(array\_dimensionable(puntero(letra)))}\\
  \tupItem{jugadores}{array\_dimensionable(tupla(
    \begin{adjustwidth}{3em}{}\
        \textrm{$puntaje$:} nat\\
        \textrm{, $mazo$:} array\_dimensionable(letra)\\
        \textrm{, $cantFichas$:} array\_dimensionable(nat)
    \end{adjustwidth}))}\\
  \tupItem{jugadorActual}{nat}\\
  \tupItem{repositorio}{cola(letra)}\\
  \tupItem{variante}{variante}
\end{adjustwidth}\ \ \ \ \ \ \
\end{Tupla}
\end{Estructura}

\subsubsection*{Invariante de Representación}
\Rep{foo}

\par\vspace*{3ex}%

\subsubsection*{Función de Abstracción}
\Abs[estr]{foo}[e]{p}{bar}

\subsubsection*{Algoritmos}
\begin{algorithm}[H]
  \defun{hacerGuia}{\In{A}{guia}, \In{parámetroInútil}{Nat}}{bool}
  \begin{algorithmic}[1]
    \State\asignar{i}{0} \Comment{esto es $\Theta$(1)}
    \State\asignar{n}{guia.\text{cantEjercicios()}} \Comment{$\bigO$(1)}
    \State\asignar{consultas}{\textsc{diccVacio}}
    \State\textsc{prepararMate()} \Comment{$\Omega$($n^n$)}
    \While{$i < n$}
    \State\textsc{pensarEjercicio(i)}
    \If{\textsc{tengoConsultas($i$)}}
      \State\textsc{escribirConsultasEjercicio($i$,$consultas$)}
    \Else
      \State\textsc{comerBizochito()}
    \EndIf
      \State\textsc{comerBizochito()}
    \EndWhile
    \For{miVariable}
         \State hacer algo
    \EndFor
    \State\Return{\textsc{vacio?}($consultas$)}
  \end{algorithmic}
\end{algorithm}

\subsection{Servicios usados}
\newpage

%%% Local Variables:
%%% mode: latex
%%% TeX-master: "main"
%%% End:

\section{Módulo Servidor}
\begin{interfaz}{\subsection{Interfaz}}
  \seExplica{Servidor}{servidor}\\\\
  \usa{\falta}\par\noindent
  \begin{operaciones}
    \InterfazFuncion{nuevoServidor}
    {\In{k}{nat}, \In{v}{variante}, \In{r}{cola(letra)}}{servidor}
    [$tama\tilde{n}o(r)\geq tama\tilde{n}oTablero(v)*tama\tilde{n}oTablero(v)+k*\#fichas(v)$]
    {$res \igobs nuevoServidor(k,v,r)$}
    [$O(N^{2}+|\Sigma|K+FK)$]
    [Dada una cantidad de jugadores y una variante de juego, se inicia un nuevo servidor y una nueva partida de juego.]
    [\falta]\\

    \noindent\InterfazFuncion{conectar}
    {\Inout{s}{servidor}}{}
    [$\lnot empez\acute{o}?(s) \land s\igobs S_{0}$]
    {$s\igobs conectarCliente(S_{0})$}
    [$O(1)$]
    [Conecta un cliente a un servidor.]
    [\falta]\\

    \noindent\InterfazFuncion{consultar}
    {\Inout{s}{servidor}, \In{cid}{nat}}{cola(notif)}
    [$cid\leq \#conectados(s) \land s\igobs S_{0}$]
    {$s\igobs consultar(S_{0},cid)\land res\igobs notificaciones(S_{0},cid)$}
    [$O(n)$, donde $n$ es la cantidad de mensajes en la cola de dicho cliente.]
    [Consulta la cola de notificaciones de un cliente (lo cual vacía dicha cola).]
    [\falta]\\

    \noindent\InterfazFuncion{recibir}
    {\Inout{s}{servidor}, \In{cid}{nat}, \In{o}{ocurrencia}}{}
    [$cid\leq \#conectados(s) \land s\igobs S_{0}$]
    {$s\igobs recibirMensaje(S_{0},cid,o)$}
    [$O(m\cdot\Lmax^{2}+|\Sigma|K+FK)$]
    [Recibe un mensaje de un cliente.]
    [\falta]\\

    \noindent\InterfazFuncion{clientesEsperados}
    {\In{s}{servidor}}{nat}
    [true]
    {$res\igobs \#esperados(s)$}
    [$O(1)$]
    [Obtiene el número de clientes esperados.]
    [\falta]\\

    \noindent\InterfazFuncion{clientesConectados}
    {\In{s}{servidor}}{nat}
    [true]
    {$res\igobs \#conectados(s)$}
    [$O(1)$]
    [Obtiene el número de clientes conectados.]
    [\falta]\\

    \noindent\InterfazFuncion{partida}
    {\In{s}{servidor}}{juego}
    [true]
    {$res\igobs juego(s)$}
    [$O(1)$]
    [Obtiene el juego que se está jugando en el servidor.]
    [\falta]\\

    \noindent\InterfazFuncion{empezó?}
    {\In{s}{servidor}}{bool}
    [true]
    {$res\igobs empez\acute{o}?(s)$}
    [$O(1)$]
    [Determina si la partida empezó.]
    [\falta]\\
  \end{operaciones}
\end{interfaz}

\subsection{Implementación}

\subsubsection{Representación}

\begin{Estructura}{servidor}[servidor\_estr]
\begin{Tupla}[servidor\_estr]
\begin{adjustwidth}{3em}{}\ \
  \tupItem{juego}{juego}\\
  \tupItem{jugadoresConectados}{nat}\\
  \tupItem{jugadoresEsperados}{nat}\\
  \tupItem{configuración}{tupla(\textrm{$variante:$} variante\textrm{$,\ repositorio:$} cola(letra))}\\
  \tupItem{notificaciones}{array\_dimensionable(cola(notif))}
\end{adjustwidth}\ \ \ \ \ \ \
\end{Tupla}
\end{Estructura}

\subsubsection{Invariante de Representación}

\subsubsection{Función de Abstracción}
\Abs[servidor\_estr]{servidor}[e]{S}{$
  \#esperados(S)\igobs e.jugadoresEsperados\land\\
  \#conectados(S)\igobs e.jugadoresConectados\land\\
  configuraci\acute{o}n(S)\igobs e.configuraci\acute{o}n\land\\
  juego(S)\igobs e.juego\yluego\\
  (\forall i:\texttt{nat})(i<\#conectados(S)\impluego notificaciones(S,i)\igobs e.notificaciones[i])
$}

\subsubsection{Algoritmos}

\begin{algorithm}[H]
    \defun{iNuevoServidor}{\In{k}{nat}, \In{v}{variante}, \In{r}{cola(letra)}}{servidor\_estr}
    \begin{algorithmic}[1]
        \State\asignar{res.juego}{nuevoJuego(k,v,r)}\Comment{$O(N^{2}+|\Sigma|K+FK)$}
        \State\asignar{res.jugadoresConectados}{0}
        \State\asignar{res.jugadoresEsperados}{k}
        \State\asignar{res.configuraci\acute{o}n}{\langle v,r\rangle}
        \State\asignar{res.notificaciones}{\textsc{CrearArreglo}(k)}
        \For{$notif\in res.notificaciones$}
            \State\asignar{notif}{\textsc{Vacía}()}
        \EndFor
        \State\Return$res$
    \end{algorithmic}
\end{algorithm}

\begin{algorithm}[H]
    \defunv{iConectar}{\Inout{s}{servidor}}
    \begin{algorithmic}[1]
        \State$\textsc{Encolar}(s.notificaciones[s.jugadoresConectados],\textsc{IdCliente}(s.jugadoresConectados))$
        \State$s.jugadoresConectados++$
        \If{$\textsc{empezó?}(s)$}
            \For{$notif\in s.notificaciones$}
                \State$\textsc{Encolar}(notif,\textsc{Empezar}(\textsc{tamañoTablero}(s.configuraci\acute{o}n.variante)))$
                \State$\textsc{Encolar}(notif,\textsc{TurnoDe}(0))$
            \EndFor
        \EndIf
    \end{algorithmic}
\end{algorithm}

\begin{algorithm}[H]
    \defun{iClientesEsperados}{\In{s}{servidor\_estr}}{nat}
    \begin{algorithmic}[1]
        \State\Return$s.jugadoresEsperados$
    \end{algorithmic}
\end{algorithm}

\begin{algorithm}[H]
    \defun{iConsultar}{\Inout{s}{servidor\_estr}, \In{cid}{nat}}{cola(notif)}
    \begin{algorithmic}[1]
        \State\asignar{res}{\textsc{Copiar}(s.notificaciones[cid])}\Comment{$\Theta(\sum_{i=1}^{n}copy(c[i]))=\Theta(\sum_{i=1}^{n}1)=\Theta(n)\in O(n)$, con $c=s.notificaciones[cid]$}
        \State\asignar{s.notificaciones[cid]}{\textsc{Vacía}()}\Comment$O(1)$
        \State\Return$res$
    \end{algorithmic}
\end{algorithm}

\begin{algorithm}[H]
    \defunv{iRecibir}{\Inout{s}{servidor\_estr}, \In{cid}{nat}, \In{o}{ocurrencia}}
    \begin{algorithmic}[1]
        \If{$\textsc{jugadaVálida?}(s.juego,o)$}
            \State\asignar{\#fichasRepuestas}{\textsc{fichasPorJugador}(s.configuraci\acute{o}n.variante)-\#o}
            \State\asignar{repoViejo}{copy(s.configuraci\acute{o}n.repositorio)}
            \State\asignar{repuestos}{\textsc{Vacío}()}\Comment{Asumimos que \texttt{multiconj(letra)} es un Diccionario Lineal}
            \For{$1\dots\#fichasRepuestas$}
                \State\asignar{ficha}{\textsc{Proximo}(repoViejo)}
                \If{$\textsc{Definido?}(repuestos,ficha)$}
                    \State$\textsc{Definir}(repuestos,ficha,\textsc{Significado}(repuestos,ficha)+1)$
                \Else
                    \State$\textsc{Definir}(repuestos,ficha,1)$
                \EndIf
                \State$\textsc{Desencolar}(repoViejo)$
            \EndFor
            \State\asignar{puntajeViejo}{\textsc{puntaje}(s.juego,cid)}
            \State$\textsc{ubicar}(s.juego,o)$
            \State\asignar{puntajeNuevo}{\textsc{puntaje}(s.juego,cid)}
            \For{$0\leq i<tam(s.notificaciones)$}
                \State\asignar{notif}{s.notificaciones[i]}
                \State$\textsc{Encolar}(notif,\textsc{Ubicar}(cid,o)$\Comment{\textsc{Ubicar} se refiere al item de notificación}
                \State$\textsc{Encolar}(notif,\textsc{SumaPuntos}(cid,puntajeNuevo-puntajeViejo)$
                \If{$i=cid$}
                    \State$\textsc{Encolar}(notif,\textsc{Reponer}(repuestos))$
                \EndIf
                \State$\textsc{Encolar}(notif,\textsc{TurnoDe}(\textsc{turno}(s.juego))$
            \EndFor
        \Else
            \State$\textsc{Encolar}(s.notificaciones[cid],\textsc{Mal})$
        \EndIf
    \end{algorithmic}
\end{algorithm}

\begin{algorithm}[H]
    \defun{iClientesConectados}{\In{s}{servidor\_estr}}{nat}
    \begin{algorithmic}[1]
        \State\Return$s.jugadoresConectados$
    \end{algorithmic}
\end{algorithm}

\begin{algorithm}[H]
    \defun{iPartida}{\In{s}{servidor\_estr}}{juego}
    \begin{algorithmic}[1]
        \State\Return$s.juego$
    \end{algorithmic}
\end{algorithm}

\begin{algorithm}[H]
    \defun{iEmpezó}{\In{s}{servidor\_estr}}{bool}
    \begin{algorithmic}[1]
        \State\Return$s.jugadoresEsperados=s.jugadoresConectados$
    \end{algorithmic}
\end{algorithm}
\newpage

%%% Local Variables:
%%% mode: latex
%%% TeX-master: "main"
%%% End:

\section{Módulos auxiliares}
\subsection{Módulo Variante}
\begin{interfaz}{\subsubsection{Interfaz}}
  \seExplica{Variante}{variante}\\
  \usa{\falta}
  \par\noindent
  \begin{operaciones}
    \InterfazFuncion{nuevaVariante}
    {
      \begin{adjustwidth}{2em}{}
      \In{n}{nat},\\
      \In{f}{nat},\\
      \In{puntajes}{dicc(letra, nat)},\\
      \In{legítimas}{conj(secu(letra))}
      \end{adjustwidth}
    }{variante}
    [$n>0\land f>0$]
    {$res \igobs nuevaVariante(n,f,puntajes,leg\acute{\imath}timas)$}
    [$O(1)$]
    [Dada una cantidad de jugadores y una variante de juego, se inicia un nuevo juego con el tablero vacío y con un repositorio de fichas acorde.]
    [\falta]
    [\falta]\\

    \noindent\InterfazFuncion{jugadaVálida?}
    {
      \In{j}{juego},
      \In{o}{occurrencia}
    }{bool}
    [true]
    {$res\igobs jugadaV\acute{a}lida?(j,o)$}
    [$O(L_{\texttt{máx}}^{2})$]
    [Determina si una jugada es válida.]
    [\falta]
    [\falta]\\

    \noindent\InterfazFuncion{ubicar}
    {
      \Inout{j}{juego},
      \In{o}{occurrencia}
    }{}
    [$jugadaV\acute{a}lida(j,o) \land j\igobs J_{0} $]
    {$j\igobs ubicar(J_{0},o)$}
    [$O(m)$, donde $m$ es el número de fichas que se ubican.]
    [Ubica un conjunto de fichas en el juego.]
    [\falta]
    [\falta]\\

    \noindent\InterfazFuncion{variante}
    {
      \In{j}{juego}
    }{variante}
    [true]
    {$res\igobs variante(j)$}
    [$O(1)$]
    [Obtiene información sobre la variante del juego.]
    [\falta]
    [\falta]\\

    \noindent\InterfazFuncion{turno}
    {
      \In{j}{juego}
    }{nat}
    [true]
    {$res\igobs turno(j)$}
    [$O(1)$]
    [Obtiene al jugador del turno actual.]
    [\falta]
    [\falta]\\

    \noindent\InterfazFuncion{puntaje}
    {
      \In{j}{juego},
      \In{i}{nat}
    }{nat}
    [$i < \#jugadores(j)$]
    {$res\igobs puntaje(j,i)$}
    [$O(1+m\cdot L_{\texttt{máx}})$, donde $m$ es la cantidad de fichas que ubicó el
jugador desde la última vez que se invocó a esta operación.]
    [Obtiene el puntaje de un jugador.]
    [\falta]
    [\falta]\\

    \noindent\InterfazFuncion{celda}
    {
      \In{J}{juego},
      \In{i}{nat},
      \In{j}{nat}
    }{puntero(letra)}
    [$enTablero?(tablero(J),i,j)$]
    {$res\igobs\IFL{hayLetra?(tablero(J),i,j)}THEN{\&letra(tablero(J),i,j)}ELSE{\NULL}FI$}
    [$O(1)$]
    [Obtiene el contenido del tablero en una coordenada $(i,j)$.]
    [\falta]
    [\falta]\\

    \noindent\InterfazFuncion{\#letraTieneJugador}
    {
      \In{j}{juego},
      \In{x}{letra},
      \In{i}{nat}
    }{nat}
    [$i < \#jugadores(j)$]
    {$res\igobs \#(x,fichas(j,i))$}
    [$O(1)$ \falta $O(F)$]
    [Dada una cierta letra $x$ del alfabeto, conocer cuántas fichas tiene un jugador de dicha letra.]
    [\falta]
    [\falta]\\

  \end{operaciones}
\end{interfaz}

\subsubsection{Implementación}
\textbf{Representación}
\begin{Estructura}{juego}[juego\_estr]
\begin{Tupla}[juego\_estr]
\begin{adjustwidth}{3em}{}\ \
  \tupItem{tablero}{array\_dimensionable(array\_dimensionable(puntero(letra)))}\\
  \tupItem{jugadores}{array\_dimensionable(tupla(\textrm{$puntaje$:} nat\textrm{, $mazo$:} array\_dimensionable(letra)))}\\
  \tupItem{jugadorActual}{nat}\\
  \tupItem{repositorio}{cola(letra)}\\
  \tupItem{variante}{variante}
\end{adjustwidth}\ \ \ \ \ \ \
\end{Tupla}
\end{Estructura}

\textbf{Invariante de Representación}
\Rep{foo}

\par\vspace*{3ex}%

\textbf{Función de Abstracción}
\Abs[estr]{foo}[e]{p}{bar}

\textbf{Algoritmos}
\begin{algorithm}[H]
  \defun{hacerGuia}{\In{A}{guia}, \In{parámetroInútil}{Nat}}{bool}
  \begin{algorithmic}[1]
    \State\asignar{i}{0} \Comment{esto es $\Theta$(1)}
    \State\asignar{n}{guia.\text{cantEjercicios()}} \Comment{$\bigO$(1)}
    \State\asignar{consultas}{\textsc{diccVacio}}
    \State\textsc{prepararMate()} \Comment{$\Omega$($n^n$)}
    \While{$i < n$}
    \State\textsc{pensarEjercicio(i)}
    \If{\textsc{tengoConsultas($i$)}}
      \State\textsc{escribirConsultasEjercicio($i$,$consultas$)}
    \Else
      \State\textsc{comerBizochito()}
    \EndIf
      \State\textsc{comerBizochito()}
    \EndWhile
    \For{miVariable}
         \State hacer algo
    \EndFor
    \State\Return{\textsc{vacio?}($consultas$)}
  \end{algorithmic}
\end{algorithm}

%%% Local Variables:
%%% mode: latex
%%% TeX-master: "main"
%%% End:


\end{document}
