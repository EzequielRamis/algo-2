\section{Módulo Juego}
\begin{interfaz}{\subsection{Interfaz}}
  \seExplica{Juego}{juego}\\\\
  \usa{Bool, Nat, Cola, Letra, Ocurrencia, Variante}
  \par\noindent
  \begin{operaciones}
    \InterfazFuncion{nuevoJuego}
    {\In{k}{nat}, \In{v}{variante}, \In{r}{cola(letra)}}{juego}
    [$tama\tilde{n}o(r)\geq tama\tilde{n}oTablero(v)*tama\tilde{n}oTablero(v)+k*\#fichas(v)\land k>0$]
    {$res \igobs nuevoJuego(k,v,r)$}
    [$O(N^{2}+|\Sigma|K+FK)$]
    [Dada una cantidad de jugadores, una variante de juego y un repositorio de fichas, se inicia un nuevo juego con el tablero vacío.]
    [\falta]\\

    \noindent\InterfazFuncion{jugadaVálida?}
    {\In{j}{juego}, \In{o}{occurrencia}}{bool}
    [true]
    {$res\igobs jugadaV\acute{a}lida?(j,o)$}
    [$O(\Lmax^{2})$]
    [Determina si una jugada es válida.]
    [\falta]\\

    \noindent\InterfazFuncion{ubicar}
    {\Inout{j}{juego}, \In{o}{occurrencia}}{}
    [$jugadaV\acute{a}lida(j,o) \land j\igobs J_{0} $]
    {$j\igobs ubicar(J_{0},o)$}
    [$O(m)$, donde $m$ es el número de fichas que se ubican.]
    [Ubica un conjunto de fichas en el tablero.]
    [\falta]\\

    \noindent\InterfazFuncion{variante}
    {\In{j}{juego}}{variante}
    [true]
    {$res\igobs variante(j)$}
    [$O(1)$]
    [Obtiene información sobre la variante del juego.]
    [\falta]\\

    \noindent\InterfazFuncion{turno}
    {\In{j}{juego}}{nat}
    [true]
    {$res\igobs turno(j)$}
    [$O(1)$]
    [Obtiene al jugador del turno actual.]
    [\falta]\\

    \noindent\InterfazFuncion{puntaje}
    {\In{j}{juego}, \In{i}{nat}}{nat}
    [$i < \#jugadores(j)$]
    {$res\igobs puntaje(j,i)$}
    [$O(1+m\cdot \Lmax)$, donde $m$ es la cantidad de fichas que ubicó el
jugador desde la última vez que se invocó a esta operación.]
    [Obtiene el puntaje de un jugador.]
    [\falta]\\

    \noindent\InterfazFuncion{enTablero?}
    {\In{J}{juego}, \In{i}{nat}, \In{j}{nat}}{bool}
    [true]
    {$res\igobs enTablero?(tablero(J),i,j)$}
    [$O(1)$]
    [Determina si una coordenada $(i,j)$ está en el rango del tablero.]
    [\falta]\\

    \noindent\InterfazFuncion{hayLetra?}
    {\In{J}{juego}, \In{i}{nat}, \In{j}{nat}}{bool}
    [$enTablero?(tablero(J),i,j)$]
    {$res\igobs hayLetra?(tablero(J),i,j)$}
    [$O(1)$]
    [Determina si una celda del tablero dada una coordenada $(i,j)$ está ocupada por una letra.]
    [\falta]\\

    \noindent\InterfazFuncion{letra}
    {\In{J}{juego}, \In{i}{nat}, \In{j}{nat}}{letra}
    [$enTablero?(tablero(J),i,j)\yluego hayLetra?(tablero(J),i,j)$]
    {$res\igobs letra(tablero(J),i,j)$}
    [$O(1)$]
    [Obtiene el contenido de una celda del tablero dada una coordenada $(i,j)$.]
    [\falta]\\

    \noindent\InterfazFuncion{\#letraTieneJugador}
    {\In{j}{juego}, \In{x}{letra}, \In{i}{nat}}{nat}
    [$i < \#jugadores(j)$]
    {$res\igobs \#(x,fichas(j,i))$}
    [$O(1)$]
    [Dada una cierta letra $x$ del alfabeto, conocer cuántas fichas tiene un jugador de dicha letra.]
    [\falta]\\

  \end{operaciones}
\end{interfaz}

\subsection{Implementación}

\subsubsection*{Representación}
\begin{Estructura}{juego}[juego\_estr]
\begin{Tupla}[juego\_estr]
\begin{adjustwidth}{3em}{}\ \
  \tupItem{tablero}{array\_dimensionable(array\_dimensionable(puntero(letra)))}\\
  \tupItem{jugadores}{array\_dimensionable(tupla(
    \begin{adjustwidth}{3em}{}\
        \textrm{$puntaje$:} nat\\
        \textrm{, $cantFichas$:} array\_dimensionable(nat)\\
        \textrm{, $cantFichasTotal$:} nat
    \end{adjustwidth}))}\\
  \tupItem{jugadorActual}{nat}\\
  \tupItem{repositorio}{cola(letra)}\\
  \tupItem{variante}{variante}
\end{adjustwidth}\ \ \ \ \ \ \
\end{Tupla}
\end{Estructura}

\subsubsection*{Invariante de Representación}
\Rep[juego\_estr]{$
  (\forall j:\texttt{jugador}\footnotemark[1])(j\in\footnotemark[2] e.jugadores\implies(
  $\begin{adjustwidth}{1em}{0em}
    $
    long(j.cantFichas)=\textsc{fichasPorJugador}(e.variante)\land\\
    long(j.cantFichas)=\textsc{dom}()\yluego\\
    \sum_{f\in j.cantFichas}f=j.cantFichasTotal
    $
  \end{adjustwidth}$))\land\\
  long(e.tablero)=\textsc{tamañoTablero}(e.variante)\land\\
  e.jugadorActual\leq long(e.jugadores)\land\\
  (\forall col:\texttt{array\_{dimensionable(puntero(letra))}})(col\in e.tablero\implies long(col)=long(e.tablero))\yluego\\
  \sum_{j\in e.jugadores}j.puntaje\geq
  \sum_{i,j<long(e.tablero)}\LIF\ e.tablero[i][j]\neq\textsc{NULL}$
  \begin{adjustwidth}{20em}{0em}
  \begin{adjustwidth}{1em}{0em}
    $\LTHEN\ \textsc{puntajeLetra}(e.variante,*e.tablero[i][j])\\
    \LELSE\ 0$
  \end{adjustwidth}
    $\LFI$
  \end{adjustwidth}
}

\footnotetext[1]{Es un renombre de $\texttt{tupla}(puntaje:\texttt{nat}, cantFichas:\texttt{array\_{dimensionable(letra)}}, cantFichasTotal:\texttt{nat})$ para simplififcar.}
\footnotetext[2]{Asumimos que existe la operación pertenece $\in$ del tipo \texttt{array\_dimensionable}.}

\par\vspace*{3ex}%

\subsubsection*{Función de Abstracción}
\Abs[juego\_estr]{juego}[e]{J}{$
  e.variante\igobs variante(J)\land\\
  e.repositorio\igobs repositorio(J)\land\\
  e.jugadorActual\igobs turno(J)\land\\
  (long(e.jugadores)\igobs \#jugadores(J)\yluego$
  \begin{adjustwidth}{11.4em}{0em}
  $(\forall i:\texttt{nat})(i<long(e.jugadores)\impluego($
  \begin{adjustwidth}{2em}{0em}
    $
    \textsc{puntaje}(e.jugadores[i])\igobs puntaje(J,i)))\land\\
    (\forall l:\texttt{letra})(e.cantFichas[\textsc{ord}(l)]=\#(l,fichas(J,i)))
    $
  \end{adjustwidth}$))\land\\
  long(e.tablero)\igobs tama\tilde{n}o(tablero(J))\yluego\\
  (\forall i,j:\texttt{nat})(enTablero?(tablero(J),i,j)\impluego($
  \begin{adjustwidth}{2em}{0em}
    $
      hayLetra?(tablero(J),i,j)\iff e.tablero[i][j]\neq\textsc{NULL}\land\\
      hayLetra?(tablero(J),i,j)\impluego letra(tablero(J),i,j)\igobs *e.tablero[i][j]
    $
  \end{adjustwidth}$))$
  \end{adjustwidth}
}

\subsubsection*{Algoritmos}
\begin{algorithm}[H]
  \defun{hacerGuia}{\In{A}{guia}, \In{parámetroInútil}{Nat}}{bool}
  \begin{algorithmic}[1]
    \State\asignar{i}{0} \Comment{esto es $\Theta$(1)}
    \State\asignar{n}{guia.\text{cantEjercicios()}} \Comment{$\bigO$(1)}
    \State\asignar{consultas}{\textsc{diccVacio}}
    \State\textsc{prepararMate()} \Comment{$\Omega$($n^n$)}
    \While{$i < n$}
    \State\textsc{pensarEjercicio(i)}
    \If{\textsc{tengoConsultas($i$)}}
      \State\textsc{escribirConsultasEjercicio($i$,$consultas$)}
    \Else
      \State\textsc{comerBizochito()}
    \EndIf
      \State\textsc{comerBizochito()}
    \EndWhile
    \For{miVariable}
         \State hacer algo
    \EndFor
    \State\Return{\textsc{vacio?}($consultas$)}
  \end{algorithmic}
\end{algorithm}

\subsection{Servicios usados}
\newpage

%%% Local Variables:
%%% mode: latex
%%% TeX-master: "main"
%%% End:
