\section{Módulo Juego}
\begin{interfaz}{\subsection{Interfaz}}
  \seExplica{Juego}{juego}\\\\
  \usa{Bool, Nat, Cola, Letra, Ocurrencia, Variante}
  \par\noindent
  \begin{operaciones}
    \InterfazFuncion{nuevoJuego}
    {\In{k}{nat}, \In{v}{variante}, \In{r}{pila(letra)}}{juego}
    [$tama\tilde{n}o(r)\geq tama\tilde{n}oTablero(v)*tama\tilde{n}oTablero(v)+k*\#fichas(v)\land k>0$]
    {$res \igobs nuevoJuego(k,v,r)$}
    [$O(N^{2}+|\Sigma|K+FK)$]
    [Dada una cantidad de jugadores, una variante de juego y un repositorio de fichas, se inicia un nuevo juego con el tablero vacío.]
    [\falta]\\

    \noindent\InterfazFuncion{jugadaVálida?}
    {\In{j}{juego}, \In{o}{occurrencia}}{bool}
    [true]
    {$res\igobs jugadaV\acute{a}lida?(j,o)$}
    [$O(\Lmax^{2})$]
    [Determina si una jugada es válida.]
    [\falta]\\

    \noindent\InterfazFuncion{ubicar}
    {\Inout{j}{juego}, \In{o}{occurrencia}}{}
    [$jugadaV\acute{a}lida(j,o) \land j\igobs J_{0} $]
    {$j\igobs ubicar(J_{0},o)$}
    [$O(m)$, donde $m$ es el número de fichas que se ubican.]
    [Ubica un conjunto de fichas en el tablero.]
    [\falta]\\

    \noindent\InterfazFuncion{variante}
    {\In{j}{juego}}{variante}
    [true]
    {$res\igobs variante(j)$}
    [$O(1)$]
    [Obtiene información sobre la variante del juego.]
    [\falta]\\

    \noindent\InterfazFuncion{turno}
    {\In{j}{juego}}{nat}
    [true]
    {$res\igobs turno(j)$}
    [$O(1)$]
    [Obtiene al jugador del turno actual.]
    [\falta]\\

    \noindent\InterfazFuncion{puntaje}
    {\In{j}{juego}, \In{i}{nat}}{nat}
    [$i < \#jugadores(j)$]
    {$res\igobs puntaje(j,i)$}
    [$O(1+m\cdot \Lmax)$, donde $m$ es la cantidad de fichas que ubicó el
jugador desde la última vez que se invocó a esta operación.]
    [Obtiene el puntaje de un jugador.]
    [\falta]\\

    \noindent\InterfazFuncion{enTablero?}
    {\In{J}{juego}, \In{i}{nat}, \In{j}{nat}}{bool}
    [true]
    {$res\igobs enTablero?(tablero(J),i,j)$}
    [$O(1)$]
    [Determina si una coordenada $(i,j)$ está en el rango del tablero.]
    [\falta]\\

    \noindent\InterfazFuncion{hayLetra?}
    {\In{J}{juego}, \In{i}{nat}, \In{j}{nat}}{bool}
    [$enTablero?(tablero(J),i,j)$]
    {$res\igobs hayLetra?(tablero(J),i,j)$}
    [$O(1)$]
    [Determina si una celda del tablero dada una coordenada $(i,j)$ está ocupada por una letra.]
    [\falta]\\

    \noindent\InterfazFuncion{letra}
    {\In{J}{juego}, \In{i}{nat}, \In{j}{nat}}{letra}
    [$enTablero?(tablero(J),i,j)\yluego hayLetra?(tablero(J),i,j)$]
    {$res\igobs letra(tablero(J),i,j)$}
    [$O(1)$]
    [Obtiene el contenido de una celda del tablero dada una coordenada $(i,j)$.]
    [\falta]\\

    \noindent\InterfazFuncion{\#letraTieneJugador}
    {\In{j}{juego}, \In{x}{letra}, \In{i}{nat}}{nat}
    [$i < \#jugadores(j)$]
    {$res\igobs \#(x,fichas(j,i))$}
    [$O(1)$]
    [Dada una cierta letra $x$ del alfabeto, conocer cuántas fichas tiene un jugador de dicha letra.]
    [\falta]\\

  \end{operaciones}
\end{interfaz}

\newpage
\subsection{Implementación}

\subsubsection*{Representación}
\begin{Estructura}{juego}[juego\_estr]
\begin{Tupla}[juego\_estr]
\begin{adjustwidth}{3em}{}\ \
  \tupItem{tablero}{tab}\\
  \tupItem{jugadores}{array\_dimensionable(jugador)}\\
  \tupItem{tiempo}{nat}\\
  \tupItem{repositorio}{pila(letra)}\\
  \tupItem{variante}{variante}
\end{adjustwidth}\ \ \ \ \ \ \
\end{Tupla}
\begin{adjustwidth}{3em}{}\
\begin{Tupla}[jugador][y]
    \begin{adjustwidth}{2em}{}\
      \texttt{
        \textrm{$puntaje$:} nat\\
        \textrm{, $historial$:} lista(tupla(\textrm{$ocurrencia$:} ocurrencia\textrm{, $tiempo$:} nat))\\
        \textrm{, $jugadasSinCalcularPuntaje$:} nat\\
        \textrm{, $cantFichasPorLetra$:} array\_dimensionable(nat)\\
        \textrm{, $cantFichasTotal$:} nat
        }
      \end{adjustwidth}
\end{Tupla}
    \end{adjustwidth}
\end{Estructura}

\subsubsection*{Invariante de Representación}
\Rep[juego\_estr]{$
  tama\tilde{n}o(e.tablero)=tama\tilde{n}oTablero(e.variante)\land\\
  % dentro de mod tablero
  % (\forall i:\texttt{nat})(i<tama\tilde{n}o(e.tablero)\impluego tam(e.tablero[i])=tam(e.tablero))\yluego\\
  % formatear con operaciones de interfaz tablero
  % (\forall i,j:\texttt{nat})((i,j<tam(e.tablero)\yluego e.tablero[i][j]\neq\textsc{NULL})\impluego e.tablero[i][j].tiempo<e.tiempo)\yluego\\
  (\forall i:\texttt{nat})(i<tam(e.jugadores)\impluego(
  $\begin{adjustwidth}{1em}{0em}
    $
    tam(e.jugadores[i].cantFichasPorLetra)=\#fichas(e.variante)\land\\
    tam(e.jugadores[i].cantFichasPorLetra)=\textsc{dom}()\yluego\\
    \sum_{f<\textsc{dom}()}e.jugadores[i].cantFichasPorLetra[f]=e.jugadores[i].cantFichasTotal\land\\
    e.jugadores[i].jugadasSinCalcularPuntaje\leq tam(e.jugadores[i].historial)\yluego\\
    tam(e.jugadores[i].historial)=\lceil e.tiempo/tam(e.jugadores)\rceil\land\\
    (\forall h:\texttt{nat})(h<long(e.jugadores[i].historial)\impluego
    $\begin{adjustwidth}{1em}{0em}
      $
      % e.jugadores[i].historial[h].tiempo = (long(e.jugadores[i].historial)-1-h)*tam(e.jugadores)+i\land\\
      e.jugadores[i].historial[h].tiempo = h*tam(e.jugadores)+i\land\\
      ocurrenciaFormaPalabra?(e.jugadores[i].historial[h].ocurrencia)
      $
    \end{adjustwidth}$)
  $
\end{adjustwidth}$))\yluego\\
    ocurrenciasV\acute{a}lidas?(nuevoTablero(tama\tilde{n}o(e.tablero)), historiales(e.jugadores,0))\yluego\\
    e.tablero\igobs ponerOcurrencias(nuevoTablero(tama\tilde{n}o(e.tablero)), historiales(e))\yluego\\
  (\forall i:\texttt{nat})(i<tam(e.jugadores)\impluego
  $\begin{adjustwidth}{1em}{0em}
    $
    % e.jugadores[i].puntaje=\sum_{k<e.jugadores[i].jugadasSinCalcularPuntaje} puntajeOcurrencia(e,i,k)
    e.jugadores[i].puntaje=\sum_{k<tam(e.jugadores[i].historial)-e.jugadores[i].jugadasSinCalcularPuntaje}
  $\begin{adjustwidth}{13em}{0em}
$puntajeOcurrencia(e,i,k)$
  \end{adjustwidth}
  \end{adjustwidth}$)
  \\\\\textbf{donde}
  $\begin{adjustwidth}{1em}{0em}
    $
    \def\dotminus{\mathbin{\ooalign{\hss\raise0.2ex\hbox{*}\hss\cr\mathsurround=0pt$-$}}}
    % historialesCalculados: \texttt{juego\_{estr}}\longrightarrow\texttt{multiconj(tupla(ocurrencia,nat))}\\
    % historialesCalculados(e')\equiv historiales(e)\dotminus historialesSinCalcular(e)\\\\
    historiales: \texttt{juego\_estr}\longrightarrow\texttt{multiconj(tupla(ocurrencia,nat))}\\
    historiales(e')\equiv historialesDesdeHastaTiempo(e'.jugadores,0,0,e.tiempo)\\
    $
  \end{adjustwidth}
}
  \begin{adjustwidth}{7.5em}{0em}
    $
    \def\dotminus{\mathbin{\ooalign{\hss\raise0.2ex\hbox{*}\hss\cr\mathsurround=0pt$-$}}}
    % historialesCalculados: \texttt{juego\_{estr}}\longrightarrow\texttt{multiconj(tupla(ocurrencia,nat))}\\
    % historialesCalculados(e')\equiv historiales(e)\dotminus historialesSinCalcular(e)\\\\
    historialesDesdeHastaTiempo: \texttt{ad(jugador)}\times\texttt{nat}\times\texttt{nat}\times\texttt{nat}\\\longrightarrow\texttt{multiconj(tupla(ocurrencia,nat))}\\
    historialesDesdeHastaTiempo(js,k,i,j)\equiv$\IFL{1em}G{k\geq tam(js)}THEN{\emptyset}ELSE{historialDesdeHastaTiempo(js[k].historial,i,j)\cup\\ historialesDesdeHastaTiempo(js,k+1,i,j)}FI$\\
    historialDesdeHastaTiempo: \texttt{lista(tupla(ocurrencia,nat))}\times\texttt{nat}\times\texttt{nat}\\\longrightarrow\texttt{multiconj(tupla(ocurrencia,nat))}\\
    historialDesdeHastaTiempo(ls,i,j)\equiv$\IFL{1em}G{vac\acute{\imath}a?(ls)}THEN{\emptyset}ELSE{historialDesdeHastaTiempo(fin(ls),i,j)\ \cup
    $\IFL{1em}G{i\leq\pi_{2}(prim(ls))<j}THEN{{prim(ls)}}ELSE{\emptyset}FI$$$}FI$\\
    ocurrenciasV\acute{a}lidas?: \texttt{tab}\times\texttt{multiconj(tupla(ocurrencia,nat))}\longrightarrow\texttt{bool}\\
    ocurrenciasV\acute{a}lidas?(t, os)\equiv$\IFL{1em}G{vac\acute\imath a?(os)}THEN{true}ELSE{celdasLibres?(t, dameUno(\pi_{1}(os)))\yluego \\ocurrenciasV\acute{a}lidas?(ponerLetras(t,dameUno(\pi_{1}(os))),sinUno(os))}FI$\\
    ponerOcurrencias: \texttt{tab}\times\texttt{multiconj(tupla(ocurrencia,nat))}\longrightarrow\texttt{tab}\\
    ponerOcurrencias(t, os)\equiv$\IFL{1em}G{vac\acute\imath a?(os)}THEN{t}ELSE{ponerOcurrencias(ponerLetras(t,dameUno(\pi_{1}(os))),sinUno(os))}FI$\\
    puntajeOcurrencia(e',i',k')\equiv \sum_{}\
  \sum_{i<tam(e.jugadores)}e.jugadores[i].puntaje=\sum_{t<e.tiempo}
  $\begin{adjustwidth}{4em}{0em}
    $\LIF\ e'.tablero[i'][j']\neq\textsc{NULL}$
    \begin{adjustwidth}{1em}{0em}
      $\LTHEN\ \textsc{puntajeLetra}(e'.variante,*e'.tablero[i'][j'].letra)\\
      \LELSE\ 0$
    \end{adjustwidth}
      $\LFI$
  \end{adjustwidth}
  \end{adjustwidth}

\par\vspace*{3ex}%

\subsubsection*{Función de Abstracción}
\Abs[juego\_estr]{juego}[e]{J}{$
  e.variante\igobs variante(J)\land\\
  e.repositorio\igobs repositorio(J)\land\\
  e.jugadorActual\igobs turno(J)\land\\
  (tam(e.jugadores)\igobs \#jugadores(J)\yluego$
  \begin{adjustwidth}{11.4em}{0em}
  $(\forall i:\texttt{nat})(i<tam(e.jugadores)\impluego($
  \begin{adjustwidth}{2em}{0em}
    $
    \textsc{puntaje}(e.jugadores[i])\igobs puntaje(J,i)))\land\\
    (\forall l:\texttt{letra})(e.cantFichasPorLetra[\textsc{ord}(l)]=\#(l,fichas(J,i)))
    $
  \end{adjustwidth}$))\land\\
  tam(e.tablero)\igobs tama\tilde{n}o(tablero(J))\yluego\\
  (\forall i,j:\texttt{nat})(enTablero?(tablero(J),i,j)\impluego($
  \begin{adjustwidth}{2em}{0em}
    $
      hayLetra?(tablero(J),i,j)\iff e.tablero[i][j]\neq\textsc{NULL}\land\\
      hayLetra?(tablero(J),i,j)\impluego letra(tablero(J),i,j)\igobs *e.tablero[i][j]
    $
  \end{adjustwidth}$))$
  \end{adjustwidth}
}

\subsubsection*{Algoritmos}
\begin{algorithm}[H]
  \defun{hacerGuia}{\In{A}{guia}, \In{parámetroInútil}{Nat}}{bool}
  \begin{algorithmic}[1]
    \State\asignar{i}{0} \Comment{esto es $\Theta$(1)}
    \State\asignar{n}{guia.\text{cantEjercicios()}} \Comment{$\bigO$(1)}
    \State\asignar{consultas}{\textsc{diccVacio}}
    \State\textsc{prepararMate()} \Comment{$\Omega$($n^n$)}
    \While{$i < n$}
    \State\textsc{pensarEjercicio(i)}
    \If{\textsc{tengoConsultas($i$)}}
      \State\textsc{escribirConsultasEjercicio($i$,$consultas$)}
    \Else
      \State\textsc{comerBizochito()}
    \EndIf
      \State\textsc{comerBizochito()}
    \EndWhile
    \For{miVariable}
         \State hacer algo
    \EndFor
    \State\Return{\textsc{vacio?}($consultas$)}
  \end{algorithmic}
\end{algorithm}

\subsection{Servicios usados}
\newpage

%%% Local Variables:
%%% mode: latex
%%% TeX-master: "main"
%%% End:
