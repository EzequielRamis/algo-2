\section{Módulo Juego}
\begin{interfaz}{\subsection{Interfaz}}
  \seExplica{Juego}{juego}\\\\
  \usa{Bool, Nat, Cola, Letra, Ocurrencia, Tablero, Variante}
  \par\noindent
  \begin{operaciones}
    \InterfazFuncion{nuevoJuego}
    {\In{k}{nat}, \In{v}{variante}, \In{r}{cola(letra)}}{juego}
    [$tama\tilde{n}o(r)\geq tama\tilde{n}oTablero(v)*tama\tilde{n}oTablero(v)+k*\#fichas(v)\land k>0$]
    {$res \igobs nuevoJuego(k,v,r)$}
    [$O(N^{2}+|\Sigma|K+FK)$]
    [Dada una cantidad de jugadores, una variante de juego y un repositorio de fichas, se inicia un nuevo juego con el tablero vacío.]
    [\falta]\\

    \noindent\InterfazFuncion{jugadaVálida?}
    {\In{j}{juego}, \In{o}{occurrencia}}{bool}
    [true]
    {$res\igobs jugadaV\acute{a}lida?(j,o)$}
    [$O(L_{\texttt{máx}}^{2})$]
    [Determina si una jugada es válida.]
    [\falta]\\

    \noindent\InterfazFuncion{ubicar}
    {\Inout{j}{juego}, \In{o}{occurrencia}}{}
    [$jugadaV\acute{a}lida(j,o) \land j\igobs J_{0} $]
    {$j\igobs ubicar(J_{0},o)$}
    [$O(m)$, donde $m$ es el número de fichas que se ubican.]
    [Ubica un conjunto de fichas en el juego.]
    [\falta]\\

    \noindent\InterfazFuncion{variante}
    {\In{j}{juego}}{variante}
    [true]
    {$res\igobs variante(j)$}
    [$O(1)$]
    [Obtiene información sobre la variante del juego.]
    [\falta]\\

    \noindent\InterfazFuncion{turno}
    {\In{j}{juego}}{nat}
    [true]
    {$res\igobs turno(j)$}
    [$O(1)$]
    [Obtiene al jugador del turno actual.]
    [\falta]\\

    \noindent\InterfazFuncion{puntaje}
    {\In{j}{juego}, \In{i}{nat}}{nat}
    [$i < \#jugadores(j)$]
    {$res\igobs puntaje(j,i)$}
    [$O(1+m\cdot L_{\texttt{máx}})$, donde $m$ es la cantidad de fichas que ubicó el
jugador desde la última vez que se invocó a esta operación.]
    [Obtiene el puntaje de un jugador.]
    [\falta]\\

    \noindent\InterfazFuncion{celdaOcupada?}
    {\In{J}{juego}, \In{i}{nat}, \In{j}{nat}}{bool}
    [$enTablero?(tablero(J),i,j)$]
    {$res\igobs hayLetra?(tablero(J),i,j)$}
    [$O(1)$]
    [Obtiene si el tablero en una coordenada $(i,j)$ está ocupado.]
    [\falta]\\

    \noindent\InterfazFuncion{celdaContenido}
    {\In{J}{juego}, \In{i}{nat}, \In{j}{nat}}{letra}
    [$enTablero?(tablero(J),i,j)\land_{L} hayLetra?(tablero(J),i,j)$]
    {$res\igobs letra(tablero(J),i,j)$}
    [$O(1)$]
    [Obtiene el contenido del tablero en una coordenada $(i,j)$ asumiendo que está ocupada.]
    [\falta]\\

    \noindent\InterfazFuncion{\#letraTieneJugador}
    {\In{j}{juego}, \In{x}{letra}, \In{i}{nat}}{nat}
    [$i < \#jugadores(j)$]
    {$res\igobs \#(x,fichas(j,i))$}
    [$O(1)$]
    [Dada una cierta letra $x$ del alfabeto, conocer cuántas fichas tiene un jugador de dicha letra.]
    [\falta]\\

  \end{operaciones}
\end{interfaz}

\subsection{Implementación}

\subsubsection*{Representación}
\begin{Estructura}{juego}[juego\_estr]
\begin{Tupla}[juego\_estr]
\begin{adjustwidth}{3em}{}\ \
  \tupItem{tablero}{array\_dimensionable(array\_dimensionable(puntero(letra)))}\\
  \tupItem{jugadores}{array\_dimensionable(tupla(
    \begin{adjustwidth}{3em}{}\
        \textrm{$puntaje$:} nat\\
        \textrm{, $mazo$:} array\_dimensionable(letra)\\
        \textrm{, $cantFichas$:} array\_dimensionable(nat)
    \end{adjustwidth}))}\\
  \tupItem{jugadorActual}{nat}\\
  \tupItem{repositorio}{cola(letra)}\\
  \tupItem{variante}{variante}
\end{adjustwidth}\ \ \ \ \ \ \
\end{Tupla}
\end{Estructura}

\subsubsection*{Invariante de Representación}
\Rep{foo}

\par\vspace*{3ex}%

\subsubsection*{Función de Abstracción}
\Abs[estr]{foo}[e]{p}{bar}

\subsubsection*{Algoritmos}
\begin{algorithm}[H]
  \defun{hacerGuia}{\In{A}{guia}, \In{parámetroInútil}{Nat}}{bool}
  \begin{algorithmic}[1]
    \State\asignar{i}{0} \Comment{esto es $\Theta$(1)}
    \State\asignar{n}{guia.\text{cantEjercicios()}} \Comment{$\bigO$(1)}
    \State\asignar{consultas}{\textsc{diccVacio}}
    \State\textsc{prepararMate()} \Comment{$\Omega$($n^n$)}
    \While{$i < n$}
    \State\textsc{pensarEjercicio(i)}
    \If{\textsc{tengoConsultas($i$)}}
      \State\textsc{escribirConsultasEjercicio($i$,$consultas$)}
    \Else
      \State\textsc{comerBizochito()}
    \EndIf
      \State\textsc{comerBizochito()}
    \EndWhile
    \For{miVariable}
         \State hacer algo
    \EndFor
    \State\Return{\textsc{vacio?}($consultas$)}
  \end{algorithmic}
\end{algorithm}

\subsection{Servicios usados}
\newpage

%%% Local Variables:
%%% mode: latex
%%% TeX-master: "main"
%%% End:
