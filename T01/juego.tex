\section{Módulo Juego}
\begin{interfaz}{\subsection{Interfaz}}
  \seExplica{Juego}{juego}\\\\
  \usa{\falta}
  \par\noindent
  \begin{operaciones}
    \InterfazFuncion{nuevoJuego}
    {\In{k}{nat}, \In{v}{variante}, \In{r}{pila(letra)}}{juego}
    [$tama\tilde{n}o(r)\geq tama\tilde{n}oTablero(v)*tama\tilde{n}oTablero(v)+k*\#fichas(v)\land k>0$]
    {$res \igobs nuevoJuego(k,v,r)$}
    [$O(N^{2}+|\Sigma|K+FK)$]
    [Dada una cantidad de jugadores, una variante de juego y un repositorio de fichas, se inicia un nuevo juego con el tablero vacío.]
    [\falta]\\

    \noindent\InterfazFuncion{jugadaVálida?}
    {\In{j}{juego}, \In{o}{occurrencia}}{bool}
    [true]
    {$res\igobs jugadaV\acute{a}lida?(j,o)$}
    [$O(\Lmax^{2})$]
    [Determina si una jugada es válida.]
    [\falta]\\

    \noindent\InterfazFuncion{ubicar}
    {\Inout{j}{juego}, \In{o}{occurrencia}}{}
    [$jugadaV\acute{a}lida(j,o) \land j\igobs J_{0} $]
    {$j\igobs ubicar(J_{0},o)$}
    [$O(m)$, donde $m$ es el número de fichas que se ubican.]
    [Ubica un conjunto de fichas en el tablero.]
    [\falta]\\

    \noindent\InterfazFuncion{variante}
    {\In{j}{juego}}{variante}
    [true]
    {$res\igobs variante(j)$}
    [$O(1)$]
    [Obtiene información sobre la variante del juego.]
    [\falta]\\

    \noindent\InterfazFuncion{turno}
    {\In{j}{juego}}{nat}
    [true]
    {$res\igobs turno(j)$}
    [$O(1)$]
    [Obtiene al jugador del turno actual.]
    [\falta]\\

    \noindent\InterfazFuncion{tiempo}
    {\In{j}{juego}}{nat}
    [true]
    {$res$ es igual a la cantidad de generadores ``ubicar'' de $j$}
    [$O(1)$]
    [Obtiene la cantidad de jugadas totales que se hicieron desde que empezó el juego.]
    [\falta]\\

    \noindent\InterfazFuncion{puntaje}
    {\In{j}{juego}, \In{i}{nat}}{nat}
    [$i < \#jugadores(j)$]
    {$res\igobs puntaje(j,i)$}
    [$O(1+m\cdot \Lmax)$, donde $m$ es la cantidad de fichas que ubicó el
jugador desde la última vez que se invocó a esta operación.]
    [Obtiene el puntaje de un jugador.]
    [\falta]\\

    \noindent\InterfazFuncion{enTablero?}
    {\In{J}{juego}, \In{i}{nat}, \In{j}{nat}}{bool}
    [true]
    {$res\igobs enTablero?(tablero(J),i,j)$}
    [$O(1)$]
    [Determina si una coordenada $(i,j)$ está en el rango del tablero.]
    [\falta]\\

    \noindent\InterfazFuncion{hayFicha?}
    {\In{J}{juego}, \In{i}{nat}, \In{j}{nat}}{bool}
    [$enTablero?(tablero(J),i,j)$]
    {$res\igobs hayLetra?(tablero(J),i,j)$}
    [$O(1)$]
    [Determina si una celda del tablero dada una coordenada $(i,j)$ está ocupada por una letra.]
    [\falta]\\

    \noindent\InterfazFuncion{ficha}
    {\In{J}{juego}, \In{i}{nat}, \In{j}{nat}}{letra}
    [$enTablero?(tablero(J),i,j)\yluego hayLetra?(tablero(J),i,j)$]
    {$res\igobs letra(tablero(J),i,j)$}
    [$O(1)$]
    [Obtiene el contenido de una celda del tablero dada una coordenada $(i,j)$.]
    [\falta]\\

    \noindent\InterfazFuncion{tiempoFicha}
    {\In{J}{juego}, \In{i}{nat}, \In{j}{nat}}{nat}
    [$enTablero?(tablero(J),i,j)\yluego hayLetra?(tablero(J),i,j)$]
    {$res$ es igual a la cantidad de generadores ``ubicar'' de j desde que empezó el juego hasta que hubo un ``ubicar'' con una ocurrencia que contenía esa coordenada.}
    [$O(1)$]
    [Obtiene el momento en que una ficha del tablero fue puesta dada una coordenada $(i,j)$.]
    [\falta]\\

    \noindent\InterfazFuncion{\#letraTieneJugador}
    {\In{j}{juego}, \In{x}{letra}, \In{i}{nat}}{nat}
    [$i < \#jugadores(j)$]
    {$res\igobs \#(x,fichas(j,i))$}
    [$O(1)$]
    [Dada una cierta letra $x$ del alfabeto, conocer cuántas fichas tiene un jugador de dicha letra.]
    [\falta]\\

  \end{operaciones}
\end{interfaz}

\newpage
\subsection{Implementación}

\subsubsection{Representación}
\begin{Estructura}{juego}[juego\_estr]
\begin{Tupla}[juego\_estr]
\begin{adjustwidth}{3em}{}\ \
  \tupItem{tablero}{array\_dimensionable(array\_dimensionable(puntero(tupla(letra,nat))))}\\
  \tupItem{jugadores}{array\_dimensionable(jugador)}\\
  \tupItem{tiempo}{nat}\\
  \tupItem{repositorio}{pila(letra)}\\
  \tupItem{variante}{variante}
\end{adjustwidth}\ \ \ \ \ \ \
\end{Tupla}
\begin{adjustwidth}{3em}{}\
\begin{Tupla}[jugador][y]
    \begin{adjustwidth}{2em}{}\
      \texttt{
        \textrm{$puntaje$:} nat\\
        \textrm{, $historial$:} lista(tupla(\textrm{$ocurrencia$:} ocurrencia\textrm{, $tiempo$:} nat))\\
        \textrm{, $historialSinVacias$:} lista(tupla(\textrm{$ocurrencia$:} ocurrencia\textrm{, $tiempo$:} nat))\\
        \textrm{, $jugadasSinCalcularPuntaje$:} nat\\
        \textrm{, $cantFichasPorLetra$:} array\_dimensionable(nat)
        }
      \end{adjustwidth}
\end{Tupla}
    \end{adjustwidth}
\end{Estructura}

\newpage
\subsubsection{Invariante de Representación}
\Rep[juego\_estr]{$
  tam(e.tablero)=tama\tilde{n}oTablero(e.variante)\land\\
  % dentro de mod tablero?
  (\forall i:\texttt{nat})(i<tam(e.tablero)\impluego tam(e.tablero[i])=tam(e.tablero))\yluego\\
  % formatear con operaciones de interfaz tablero
  (\forall i,j:\texttt{nat})((i,j<tam(e.tablero)\yluego e.tablero[i][j]\neq\textsc{NULL})\impluego e.tablero[i][j].tiempo<e.tiempo)\yluego\\
  (\forall i:\texttt{nat})(i<tam(e.jugadores)\impluego(
  $\begin{adjustwidth}{1em}{0em}
    $
    tam(e.jugadores[i].cantFichasPorLetra)=\textsc{dom}()\yluego\\
    \sum_{f<\textsc{dom}()}e.jugadores[i].cantFichasPorLetra[f]=\#fichas(e.variante)\land\\
    tam(e.jugadores[i].historial)=\lceil e.tiempo/tam(e.jugadores)\rceil\land\\
    (\forall h:\texttt{nat})(h<long(e.jugadores[i].historial)\impluego
    $\begin{adjustwidth}{1em}{0em}
      $
      % e.jugadores[i].historial[h].tiempo = (long(e.jugadores[i].historial)-1-h)*tam(e.jugadores)+i\land\\
      e.jugadores[i].historial[h].tiempo=h*tam(e.jugadores)+i\land\\
      (\forall p,q:\texttt{nat})(\forall l,l':\texttt{letra})(
    $\begin{adjustwidth}{1em}{0em}$
      \{\langle p,q,l\rangle,\langle p,q,l'\rangle\}\subseteq e.jugadores[i].historial[h].ocurrencia\implies l=l'
      $\end{adjustwidth}$)\yluego\\
      e.jugadores[i].historialSinVacias\igobs historialSinVac\acute{\imath}as(e.jugadores[i].historial,<>)\land\\
      e.jugadores[i].jugadasSinCalcularPuntaje\leq tam(e.jugadores[i].historialSinVac\acute{\imath}as)
      $
    \end{adjustwidth}$)
  $
\end{adjustwidth}$))\yluego\\
    ocurrenciasV\acute{a}lidas?(nuevoTablero(tama\tilde{n}o(e.tablero)), historiales(e.jugadores,0))\yluego\\
    e.tablero\igobs ponerOcurrencias(nuevoTablero(tama\tilde{n}o(e.tablero)), historiales(e))\yluego\\
  (\forall i:\texttt{nat})(i<tam(e.jugadores)\impluego
  $\begin{adjustwidth}{1em}{0em}
    $
    % e.jugadores[i].puntaje=\sum_{k<e.jugadores[i].jugadasSinCalcularPuntaje} puntajeOcurrencia(e,i,k)
    e.jugadores[i].puntaje=\sum_{k<tam(e.jugadores[i].historial)-e.jugadores[i].jugadasSinCalcularPuntaje}
  $\begin{adjustwidth}{13em}{0em}
$puntajeDeOcurrenciaEnTiempo(e,i,k)$
  \end{adjustwidth}
  \end{adjustwidth}$)
  \\\\\textbf{donde}
  $\begin{adjustwidth}{1em}{0em}
    $
    \def\dotminus{\mathbin{\ooalign{\hss\raise0.2ex\hbox{*}\hss\cr\mathsurround=0pt$-$}}}
    % historialesCalculados: \texttt{juego\_{estr}}\longrightarrow\texttt{multiconj(tupla(ocurrencia,nat))}\\
    % historialesCalculados(e')\equiv historiales(e)\dotminus historialesSinCalcular(e)\\\\
    historialSinVac\acute{\imath}as: \texttt{lista(tupla(ocurrencia,nat))}\longrightarrow\texttt{lista(tupla(ocurrencia,nat))}\\
    historialSinVac\acute{\imath}as(hcv,hsv)\equiv$\IFL{1em}G{vac\acute{\imath}a?(hcv)}THEN{hsv}ELSE{
      $\IFL{1em}G{vac\acute{\imath}o?(\pi_{1}(prim(hcv)))}THEN{historialSinVac\acute{\imath}as(fin(hcv),hsv)}ELSE{historialSinVac\acute{\imath}as(fin(hcv),prim(hcv)\puntito hsv)}FI$
    }FI$\\\\
    historiales: \texttt{juego\_estr}\longrightarrow\texttt{multiconj(ocurrencia)}\\
    historiales(e')\equiv historialesHastaTiempo(e'.jugadores,0,e.tiempo)\\
    $
  \end{adjustwidth}
}
  \begin{adjustwidth}{7.5em}{0em}
    $
    \def\dotminus{\mathbin{\ooalign{\hss\raise0.2ex\hbox{*}\hss\cr\mathsurround=0pt$-$}}}
    historialesHastaTiempo: \texttt{ad(jugador)}\times\texttt{nat}\times\texttt{nat}\\\longrightarrow\texttt{multiconj(ocurrencia)}\\
    historialesHastaTiempo(js,k,t)\equiv$\IFL{1em}G{k\geq tam(js)}THEN{\emptyset}ELSE{historialHastaTiempo(js,k,t)\\\cup historialesHastaTiempo(js,k+1,t)}FI$\\
    historialHastaTiempo: \texttt{ad(jugador)}\times\texttt{nat}\times\texttt{nat}\\\longrightarrow\texttt{multiconj(ocurrencia)}\\
    historialHastaTiempo(js,k,t)\equiv historialHastaTiempo'(js[k].historial,t)\\\\
    historialHastaTiempo': \texttt{lista(tupla(ocurrencia,nat))}\times\texttt{nat}\\\longrightarrow\texttt{multiconj(ocurrencia)}\\
    historialHastaTiempo'(ls,t)\equiv$\IFL{1em}G{vac\acute{\imath}a?(ls)}THEN{\emptyset}ELSE{historialHastaTiempo'(fin(ls),t)\ \cup
    $\IFL{1em}G{\pi_{2}(prim(ls))<t}THEN{{prim(ls)}}ELSE{\emptyset}FI$$$}FI$\\
    ocurrenciasV\acute{a}lidas?: \texttt{tab}\times\texttt{multiconj(ocurrencia))}\longrightarrow\texttt{bool}\\
    ocurrenciasV\acute{a}lidas?(t, os)\equiv$\IFL{1em}G{vac\acute\imath a?(os)}THEN{true}ELSE{celdasLibres?(t, dameUno(os))\yluego \\ocurrenciasV\acute{a}lidas?(ponerLetras(t,dameUno(os)),sinUno(os))}FI$\\
    ponerOcurrencias: \texttt{tab}\times\texttt{multiconj(ocurrencia)}\longrightarrow\texttt{tab}\\
    ponerOcurrencias(t, os)\equiv$\IFL{1em}G{vac\acute\imath a?(os)}THEN{t}ELSE{ponerOcurrencias(ponerLetras(t,dameUno(os)),sinUno(os))}FI$\\
    puntajeDeOcurrenciaEnTiempo: \texttt{estr\_juego}\times\texttt{nat}\times\texttt{nat}\longrightarrow\texttt{nat}\\
    puntajeDeOcurrenciaEnTiempo(e,i,k)\equiv$
    \begin{adjustwidth}{1em}{0em}$
      puntajePalabrasEstr(e.variante,t',\\palabrasUbicadas(ocurrenciasDePalabras(t'),e.jugadores[i].historial[k].ocurrencia))
    $\end{adjustwidth}
    \begin{adjustwidth}{1em}{0em}
\textbf{donde}
    \begin{adjustwidth}{1em}{0em}
      $
      tiempo\equiv e.jugadores[i].historial[k].tiempo\\
      t'\equiv ponerOcurrencias(nuevoTablero(tama\tilde{n}o(e.tablero)),
      $\IFL{1em}G{tiempo=0?}THEN{\emptyset}ELSE{historialesHastaTiempo(e.jugadores,0,tiempo-1)}FI$
      \cup\ historialHastaTiempo(e.jugadores,i,tiempo)\\
      \ \ \ \cup\ \{e.jugadores[i].historial[k].ocurrencia\})
      $
  \end{adjustwidth}
  \end{adjustwidth}
  $
  puntajePalabrasEstr: \texttt{variante}\times\texttt{tab}\times\texttt{conj(ocurrencia)}\longrightarrow\texttt{nat}\\
  puntajePalabrasEstr(v,t,os)\equiv$\IFL{1em}G{vac\acute{\imath}o?(os)}THEN{0}ELSE{puntajePalabraEstr(v,t,dameUno(os))\\+puntajePalabras(v,t,sinUno(os))}FI$\\\\
  puntajePalabraEstr: \texttt{variante}\times\texttt{tab}\times\texttt{ocurrencia}\longrightarrow\texttt{nat}\\
  puntajePalabraEstr(v,t,o)\equiv$\IFL{1em}G{vac\acute{\imath}a?(o)}THEN{0}ELSE{puntajeLetra(v,\pi_{3}(dameUno(o)))\\+puntajePalabra(v,t,sinUno(o))}FI$
  $
  \end{adjustwidth}
\par\vspace*{3ex}%

\subsubsection{Función de Abstracción}
\Abs[juego\_estr]{juego}[e]{J}{$
  e.variante\igobs variante(J)\land\\
  e.repositorio\igobs repositorio(J)\land\\
  e.tiempo\equiv turno(J)\ (mod\ \#jugadores(J))\land\\
  tam(e.tablero)\igobs tama\tilde{n}o(T)\yluego\\
  (\forall i,j:\texttt{nat})((enTablero?(T,i,j)\yluego hayLetra?(T,i,j))\impluego
  $\begin{adjustwidth}{12em}{0em}$
  (e.tablero[i][j]\neq NULL\yluego letra(T,i,j)\igobs\pi_{1}(*e.tablero[i][j])))\land
  $\end{adjustwidth}$\\
  (tam(e.jugadores)\igobs \#jugadores(J)\yluego$
  \begin{adjustwidth}{11.4em}{0em}
  $(\forall i:\texttt{nat})(i<tam(e.jugadores)\impluego($
  \begin{adjustwidth}{2em}{0em}
    $
    e.jugadores[i].puntaje\igobs puntaje(J,i)\land\\
    (\forall l:\texttt{letra})(e.cantFichasPorLetra[\textsc{ord}(l)]=\#(l,fichas(J,i)))
    $
  \end{adjustwidth}$))$
  \end{adjustwidth}
}

\subsubsection{Algoritmos}
\begin{algorithm}[H]
  \defun{iNuevoJuego}{\In{k}{nat}, \In{v}{variante}, \In{r}{cola(letra)}}{juego\_estr}
  \begin{algorithmic}[1]
    \State\asignar{res.variante}{v}
    \State\asignar{res.repositorio}{r}
    \State\asignar{res.tiempo}{0}
    \State\asignar{filas}{\textsc{CrearArreglo}(n)}
    \For{$columnas\in filas$}\Comment{$O(N)$}
        \State\asignar{columnas}{\textsc{CrearArreglo}(n)}
        \For{$celda\in columnas$}\Comment{$O(N)$}
          \State\asignar{celda}{\textsc{NULL}}
        \EndFor
    \EndFor
    \State\asignar{res.jugadores}{\textsc{CrearArreglo}(k)}
    \For{$jugador\in res.jugadores$}\Comment{$O(K)$}
        \State\asignar{jugador.puntaje}{0}
        \State\asignar{jugador.historial}{\textsc{Vacía()}}
        \State\asignar{jugador.tiemposDeJugadasVacias}{\textsc{Vacía()}}
        \State\asignar{jugador.jugadasSinCalcularPuntaje}{0}
        \State // Por cada jugador le damos su primer mazo de fichas del repositorio
        \State\asignar{jugador.cantFichasPorLetra}{\textsc{CrearArreglo}(\textsc{dom}())}
        \For{$cant\in jugador.cantFichasPorLetra$}\Comment{$O(|\Sigma|)$}
            \State\asignar{cant}{0}
        \EndFor
        \For{$1\dots\textsc{fichasPorJugador}(v)$}\Comment{$O(F)$}
            \State\asignar{ficha}{\textsc{Desapilar}(res.repositorio)}
            \State$jugador.cantFichasPorLetra[\textsc{ord}(ficha)]++$
        \EndFor
    \EndFor
    \State\Return{res}
  \end{algorithmic}
\end{algorithm}
Justificación de complejidad:
\begin{align*}
  O(N^{2})+O(K)*(O(|\Sigma|)+O(F))
    & = O(N^{2})+O(K)*O(|\Sigma|+F)\\
    & = O(N^{2})+O(K*(|\Sigma|+F))\\
    & = O(N^{2})+O(|\Sigma|K+FK)\\
    & = O(N^{2}+|\Sigma|K+FK)\\
\end{align*}

\begin{algorithm}[H]
  \defun{iJugadaVálida?}{\In{e}{estr\_juego}, \In{o}{ocurrencia}}{bool}
  \begin{algorithmic}[1]
    \If{$\textsc{Cardinal}(o)>\textsc{longPalabraMásLarga}(e.variante)$}\Comment{Con este if evitamos acotar por $m$}
    \State\Return false
    \EndIf
    \For{$ $\asignar{oIt}{\textsc{CrearIt}(o)}$;\ \textsc{HaySiguiente}(oIt);\ \textsc{Avanzar}(oIt)$}\Comment{$O(\Lmax)$}
        \State\asignar{ficha}{\textsc{Siguiente}(oIt)}
        \If{$\lnot\textsc{enTablero?}(j,\pi_{1}(ficha),\pi_{2}(ficha))\oluego\textsc{hayLetra?}(e,\pi_{1}(ficha),\pi_{2}(ficha))$}
        \State\Return false
        \EndIf
    \EndFor
    \If{$\textsc{haySuperpuestas?}(o)\lor\lnot(\textsc{esHorizontal?}(o)\lor\textsc{esVertical?}(o))$}\Comment{$O(\Lmax^2)$}
    \State\Return false
    \EndIf
    \State // Ponemos las fichas de la ocurrencia para validar
    \State$\textsc{ponerLetras}(e,o)$
    \If{$\textsc{esHorizontal?}(o)$}
      \State // Elegimos cualquier ficha y expandimos para atrás con i y para adelante con j para obtener toda la palabra horizontal
      \State\asignar{cualquierFicha}{\textsc{Siguiente}(\textsc{CrearIt}(o))}
      \State\asignar{rango}{rangoDePalabraHorizontal(e,cualquierFicha)}
      \State // Ver que todas las fichas de la ocurrencia estén incluidas en el rango, sino sacamos las letras del tablero y devolvemos false
      \For{$ $\asignar{oIt}{\textsc{CrearIt}(o)}$;\ \textsc{HaySiguiente}(oIt);\ \textsc{Avanzar}(oIt)$}\Comment{$O(\Lmax)$}
          \State\asignar{ficha}{\textsc{Siguiente}(oIt)}
          \If{$\lnot(\pi_{1}(rango)\leq\pi_{2}(ficha)\leq\pi_{2}(rango))$}
            \State$\textsc{sacarLetras}(e,o)$
            \State\Return false
          \EndIf
      \EndFor
      \If{$\lnot\textsc{formaPalabraLegítima?}(e,rango,true,\pi_{1}(cualquierFicha))$}\Comment$O(\Lmax)$
        \State$\textsc{sacarLetras}(e,o)$
        \State\Return false
      \EndIf
      \State // Vemos las palabras que se forman en las columnas
      \For{$ $\asignar{oIt}{\textsc{CrearIt}(o)}$;\ \textsc{HaySiguiente}(oIt);\ \textsc{Avanzar}(oIt)$}\Comment{$O(\Lmax)$}
        \State\asignar{ficha}{\textsc{Siguiente}(\textsc{CrearIt}(o))}
        \State\asignar{rango}{rangoDePalabraVertical(e,ficha)}
        \State // Si se forman nuevas palabras en las columnas ver que sean legítimas
        \If{$\pi_{1}(rango)\neq\pi_{2}(rango)\yluego\lnot\textsc{formaPalabraLegítima?}(e,rango,false,\pi_{2}(ficha))$}\Comment$O(\Lmax)$
          \State$\textsc{sacarLetras}(e,o)$
          \State\Return false
        \EndIf
      \EndFor

    \Else

      \State // Es el mismo código del branch true pero ahora la ocurencia es vertical
      \State\asignar{cualquierFicha}{\textsc{Siguiente}(\textsc{CrearIt}(o))}
      \State\asignar{rango}{rangoDePalabraVertical(e,cualquierFicha)}
      \For{$ $\asignar{oIt}{\textsc{CrearIt}(o)}$;\ \textsc{HaySiguiente}(oIt);\ \textsc{Avanzar}(oIt)$}\Comment{$O(\Lmax)$}
          \State\asignar{ficha}{\textsc{Siguiente}(oIt)}
          \If{$\lnot(\pi_{1}(rango)\leq\pi_{1}(ficha)\leq\pi_{2}(rango))$}
            \State$\textsc{sacarLetras}(e,o)$
            \State\Return false
          \EndIf
      \EndFor
      \If{$\lnot\textsc{formaPalabraLegítima?}(e,rango,false,\pi_{2}(cualquierFicha))$}\Comment$O(\Lmax)$
        \State$\textsc{sacarLetras}(e,o)$
        \State\Return false
      \EndIf
      \State // Vemos las palabras que se forman en las filas
      \For{$ $\asignar{oIt}{\textsc{CrearIt}(o)}$;\ \textsc{HaySiguiente}(oIt);\ \textsc{Avanzar}(oIt)$}\Comment{$O(\Lmax)$}
        \State\asignar{ficha}{\textsc{Siguiente}(\textsc{CrearIt}(o))}
        \State\asignar{rango}{rangoDePalabraHorizontal(e,ficha)}
        \State // Si se forman nuevas palabras en las filas ver que sean legítimas
        \If{$\pi_{1}(rango)\neq\pi_{2}(rango)\yluego\lnot\textsc{formaPalabraLegítima?}(e,rango,true,\pi_{1}(ficha))$}\Comment$O(\Lmax)$
          \State$\textsc{sacarLetras}(e,o)$
          \State\Return false
        \EndIf
      \EndFor


    \EndIf

    \State // Sacamos las fichas de la ocurrencia para no modificar el tablero
    \State$\textsc{sacarLetras}(j,o)$
    \State\Return true
  \end{algorithmic}
\end{algorithm}

\begin{algorithm}[H]
  \defunv{iUbicar}{\Inout{j}{estr\_juego}, \In{o}{ocurrencia}}
  \begin{algorithmic}[1]
    \State\asignar{jugador}{j.jugadores[\textsc{turno}(j)]}
    \For{$ $\asignar{oIt}{\textsc{CrearIt}(o)}$;\ \textsc{HaySiguiente}(oIt);\ \textsc{Avanzar}(oIt)$}\Comment{$O(m)$}
      \State\asignar{ficha}{\textsc{Siguiente}(oIt)}
      \State\asignar{e.tablero[\pi_{1}(ficha)][\pi_{2}(ficha)]}{\&\langle\pi_{3}(ficha),e.tiempo\rangle}
      \State$jugador.cantFichasPorLetra[\textsc{ord}(\pi_{3}(ficha))]--$
      \State$jugador.cantFichasPorLetra[\textsc{ord}(\textsc{Desapilar}(j.repositorio))]++$\Comment{$O(1)$}
    \EndFor
    \If{$\textsc{Cardinal}(o)\neq 0$}
      \State$\textsc{AgregarAtras}(jugador.historialSinVac\acute{\imath}a,\langle o, j.tiempo\rangle)$
      \State$jugador.jugadasSinCalcularPuntaje++$
    \EndIf
    \State$\textsc{AgregarAtras}(jugador.historial,\langle o, j.tiempo\rangle)$
    \State$j.tiempo++$
  \end{algorithmic}
\end{algorithm}

\begin{algorithm}[H]
  \defun{iVariante}{\In{j}{estr\_juego}}{variante}
  \begin{algorithmic}[1]
    \State\Return{j.variante}
  \end{algorithmic}
\end{algorithm}

\begin{algorithm}[H]
  \defun{iTurno}{\In{j}{estr\_juego}}{nat}
  \begin{algorithmic}[1]
    \State\Return{$j.tiempo\ \%\ tam(j.jugadores)$}
  \end{algorithmic}
\end{algorithm}

\begin{algorithm}[H]
  \defun{iTiempo}{\In{j}{estr\_juego}}{nat}
  \begin{algorithmic}[1]
    \State\Return{$j.tiempo$}
  \end{algorithmic}
\end{algorithm}

\begin{algorithm}[H]
  \defun{iPuntaje}{\In{e}{estr\_juego}, \In{i}{nat}}{nat}
  \begin{algorithmic}[1]
    \State\asignar{k}{\&e.jugadores[i].jugadasSinCalcularPuntaje}
    \State\asignar{jugador}{e.jugadores[i]}
    \State // Usamos el historial sin ocurrencias vacías para evitar acotar por su cantidad con vacías
    \State\asignar{histIt}{\textsc{CrearItUlt}(jugador.historialSinVac\acute{\imath}as)}
    \While{$*k>0$}\Comment{Todo el ciclo está acotado por $O(m\cdot\Lmax)$ porque usamos ni mas ni menos que las $m$ fichas}
      \State\asignar{jugada}{\textsc{Anterior}(histIt)}
      \State\asignar{ocIt}{\textsc{CrearIt}(jugada.ocurrencia)}
      \State\asignar{esHorizontal}{true}
      \State\asignar{ficha}{\textsc{Siguiente}(ocIt)}
      \If{\textsc{HaySiguiente?}(\textsc{Avanzar}(ocIt))}
        \State\asignar{esHorizontal}{\pi_{1}(ficha)=\pi_{1}(\textsc{Siguiente}(ocIt))}
      \EndIf
      \If{$esHorizontal$}
        \State // Sumamos las fichas de la palabra alineada horizontalmente en $O(\Lmax)$
        \State\asignar{fila}{\pi_{1}(ficha)}
        \State\asignar{i}{\pi_{2}(ficha)}
        \State\asignar{j}{\pi_{2}(ficha)+1}
        \While{$\textsc{enTablero?}(e,fila,i)\yluego\textsc{hayLetra?}(e,fila,i)\yluego\textsc{fichaTiempo}(e,fila,i)\leq jugada.tiempo$}
          \State$jugador.puntaje\ +=\ \textsc{puntajeLetra}(e.variante,\textsc{ficha}(e,fila,i))$
          \State$i--$
        \EndWhile
        \While{$\textsc{enTablero?}(e,fila,j)\yluego\textsc{hayLetra?}(e,fila,j)\yluego\textsc{fichaTiempo}(e,fila,j)\leq jugada.tiempo$}
          \State$jugador.puntaje\ +=\ \textsc{puntajeLetra}(e.variante,\textsc{ficha}(e,fila,j))$
          \State$j++$
        \EndWhile

        \State // Sumamos el resto de las fichas de las palabras cruzadas verticalmente en $O(\#jugada.ocurrencia\cdot\Lmax)$
        \For{$ $\asignar{ocIt}{\textsc{CrearIt}(jugada.ocurrencia)}$;\ \textsc{HaySiguiente}(ocIt);\ \textsc{Avanzar}(ocIt)$}
          \State\asignar{col}{\pi_{2}(ficha)}
          \State\asignar{i}{\pi_{1}(ficha)-1}\Comment{Así evitamos sumar de vuelta la ficha}
          \State\asignar{j}{\pi_{1}(ficha)+1}

          \While{$\textsc{enTablero?}(e,i,col)\yluego\textsc{hayLetra?}(e,i,col)\yluego\textsc{fichaTiempo}(e,i,col)\leq jugada.tiempo$}
            \State$jugador.puntaje\ +=\ \textsc{puntajeLetra}(e.variante,\textsc{ficha}(e,i,col))$
            \State$i--$
          \EndWhile
          \While{$\textsc{enTablero?}(e,j,col)\yluego\textsc{hayLetra?}(e,j,col)\yluego\textsc{fichaTiempo}(e,j,col)\leq jugada.tiempo$}
            \State$jugador.puntaje\ +=\ \textsc{puntajeLetra}(e.variante,\textsc{ficha}(e,j,col))$
            \State$j++$
          \EndWhile
        \EndFor

      \Else

        \State // Lo mismo que el branch true pero verticalmente
        \State\asignar{col}{\pi_{2}(ficha)}
        \State\asignar{i}{\pi_{1}(ficha)}
        \State\asignar{j}{\pi_{1}(ficha)+1}
        \While{$\textsc{enTablero?}(e,i,col)\yluego\textsc{hayLetra?}(e,i,col)\yluego\textsc{fichaTiempo}(e,i,col)\leq jugada.tiempo$}
          \State$jugador.puntaje\ +=\ \textsc{puntajeLetra}(e.variante,\textsc{ficha}(e,i,col))$
          \State$i--$
        \EndWhile
        \While{$\textsc{enTablero?}(e,j,col)\yluego\textsc{hayLetra?}(e,j,col)\yluego\textsc{fichaTiempo}(e,j,col)\leq jugada.tiempo$}
          \State$jugador.puntaje\ +=\ \textsc{puntajeLetra}(e.variante,\textsc{ficha}(e,j,col))$
          \State$j++$
        \EndWhile

        \For{$ $\asignar{ocIt}{\textsc{CrearIt}(jugada.ocurrencia)}$;\ \textsc{HaySiguiente}(ocIt);\ \textsc{Avanzar}(ocIt)$}
          \State\asignar{fil}{\pi_{1}(ficha)}
          \State\asignar{i}{\pi_{2}(ficha)-1}\Comment{Así evitamos sumar de vuelta la ficha}
          \State\asignar{j}{\pi_{2}(ficha)+1}

          \While{$\textsc{enTablero?}(e,fil,i)\yluego\textsc{hayLetra?}(e,fil,i)\yluego\textsc{fichaTiempo}(e,fil,i)\leq jugada.tiempo$}
            \State$jugador.puntaje\ +=\ \textsc{puntajeLetra}(e.variante,\textsc{ficha}(e,fil,i))$
            \State$i--$
          \EndWhile
          \While{$\textsc{enTablero?}(e,fil,j)\yluego\textsc{hayLetra?}(e,fil,j)\yluego\textsc{fichaTiempo}(e,fil,j)\leq jugada.tiempo$}
            \State$jugador.puntaje\ +=\ \textsc{puntajeLetra}(e.variante,\textsc{ficha}(e,fil,j))$
            \State$j++$
          \EndWhile
        \EndFor


      \EndIf


      \State$\textsc{Retroceder}(histIt)$
      \State$*k--$\Comment{$jugadasSinCalcularPuntaje$ se va a cero, como se espera}
    \EndWhile
    \State\Return{$e.jugadores[i].puntaje$}
  \end{algorithmic}
\end{algorithm}

\begin{algorithm}[H]
  \defun{iEnTablero?}{\In{j}{estr\_juego}, \In{i}{nat}, \In{j}{nat}}{bool}
  \begin{algorithmic}[1]
    \State\Return{$i<tam(t)\land j<tam(t)$}
  \end{algorithmic}
\end{algorithm}

\begin{algorithm}[H]
  \defun{iHayLetra?}{\In{j}{estr\_juego}, \In{i}{nat}, \In{j}{nat}}{bool}
  \begin{algorithmic}[1]
    \State\Return{$t[i][j]\neq\textsc{NULL}$}
  \end{algorithmic}
\end{algorithm}

\begin{algorithm}[H]
  \defun{iFicha}{\In{j}{estr\_juego}, \In{i}{nat}, \In{j}{nat}}{letra}
  \begin{algorithmic}[1]
    \State\Return{$\pi_{1}(*t[i][j])$}
  \end{algorithmic}
\end{algorithm}

\begin{algorithm}[H]
  \defun{iFichaTiempo}{\In{j}{estr\_juego}, \In{i}{nat}, \In{j}{nat}}{nat}
  \begin{algorithmic}[1]
    \State\Return{$\pi_{2}(*t[i][j])$}
    \State\Return{$\textsc{letra}(j.tablero,i,j)$}
  \end{algorithmic}
\end{algorithm}

\begin{algorithm}[H]
  \defun{i\#letraTieneJugador}{\In{j}{estr\_juego}, \In{l}{letra}, \In{i}{nat}}{nat}
  \begin{algorithmic}[1]
    \State\Return{$j.jugadores[i].cantFichasPorLetra[\textsc{ord}(l)]$}
  \end{algorithmic}
\end{algorithm}

\subsubsection{Algoritmos Auxiliares}
\begin{algorithm}[H]
  \defunv{ponerLetras}{\Inout{j}{estr\_juego}, \In{o}{ocurrencia}}
  \begin{algorithmic}[1]
    \For{$ $\asignar{oIt}{\textsc{CrearIt}(o)}$;\ \textsc{HaySiguiente}(oIt);\ \textsc{Avanzar}(oIt)$}\Comment{$O(\#o)$}
      \State\asignar{ficha}{\textsc{Siguiente}(oIt)}
      \State\asignar{e.tablero[\pi_{1}(ficha)][\pi_{2}(ficha)]}{\&\langle\pi_{3}(ficha),e.tiempo\rangle}
    \EndFor
  \end{algorithmic}
\end{algorithm}
\begin{algorithm}[H]
  \defunv{sacarLetras}{\Inout{j}{estr\_juego}, \In{o}{ocurrencia}}
  \begin{algorithmic}[1]
    \For{$ $\asignar{oIt}{\textsc{CrearIt}(o)}$;\ \textsc{HaySiguiente}(oIt);\ \textsc{Avanzar}(oIt)$}\Comment{$O(\#o)$}
      \State\asignar{ficha}{\textsc{Siguiente}(oIt)}
      \State\asignar{e.tablero[\pi_{1}(ficha)][\pi_{2}(ficha)]}{\textsc{NULL}}
    \EndFor
  \end{algorithmic}
\end{algorithm}
\begin{algorithm}[H]
  \defun{rangoDePalabraHorizontal}{\In{e}{estr\_juego}, \In{ficha}{tupla(nat,nat,letra)}}{tupla(nat,nat)}
  \begin{algorithmic}[1]
    \State\asignar{fila}{\pi_{1}(ficha)}
    \State\asignar{i}{\pi_{2}(ficha)}
    \State\asignar{j}{\pi_{2}(ficha)}
    \While{$\textsc{enTablero?}(e,fila,i)\yluego\textsc{hayFicha?}(e,fila,i)$}
      \State$i--$
    \EndWhile
    \While{$\textsc{enTablero?}(e,fila,j)\yluego\textsc{hayFicha?}(e,fila,j)$}
      \State$j++$
    \EndWhile
    \State\Return$\langle i,j\rangle$
  \end{algorithmic}
\end{algorithm}
\begin{algorithm}[H]
  \defun{rangoDePalabraVertical}{\In{e}{estr\_juego}, \In{ficha}{tupla(nat,nat,letra)}}{tupla(nat,nat)}
  \begin{algorithmic}[1]
    \State\asignar{columna}{\pi_{2}(ficha)}
    \State\asignar{i}{\pi_{1}(ficha)}
    \State\asignar{j}{\pi_{1}(ficha)}
    \While{$\textsc{enTablero?}(e,i,columna)\yluego\textsc{hayFicha?}(e,i,columna)$}
      \State$i--$
    \EndWhile
    \While{$\textsc{enTablero?}(e,j,columna)\yluego\textsc{hayFicha?}(e,j,columna)$}
      \State$j++$
    \EndWhile
    \State\Return$\langle i,j\rangle$
  \end{algorithmic}
\end{algorithm}
\begin{algorithm}[H]
  \defun{formaPalabraLegítima?}{\In{e}{estr\_juego}, \In{r}{tupla(nat,nat)}, \In{horizontal}{bool}, \In{padding}{nat}}{bool}
  \begin{algorithmic}[1]
    \State // Hacemos un pseudo counting sort para tener la palabra en $O(\#o)$
    \State\asignar{\langle i,j\rangle}{r}
    \State\asignar{palabra}{\textsc{CrearArreglo}(j-i)}\Comment Es \texttt{arreglo\_dimensionable(letra)}
    \If{$horizontal$}
      \For{$i\leq k\leq j$}
        \State\asignar{palabra[k-i]}{\textsc{ficha}(e,padding,k)}
      \EndFor
    \Else
      \For{$i\leq k\leq j$}
        \State\asignar{palabra[k-i]}{\textsc{ficha}(e,k,padding)}
      \EndFor
    \EndIf
    \State\asignar{palabra'}{\textsc{Vacía}()}\Comment Lista Enlazada
    \For{$0\leq k<tam(palabra)$}
      \State$\textsc{AgregarAtras}(palabra',palabra[k])$
    \EndFor
    \State\Return{$\textsc{palabraLegítima?}(e.variante,palabra')$}\Comment$O(\Lmax)$
  \end{algorithmic}
\end{algorithm}
\subsection{Servicios usados}
\newpage

%%% Local Variables:
%%% mode: latex
%%% TeX-master: "main"
%%% End:
